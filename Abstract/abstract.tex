%\newgeometry{top=2cm, left=2.7cm, right=3cm, bottom=1.45cm}
% Thesis Abstract -----------------------------------------------------
%\begin{abstractslong}   % uncommenting this line, gives a different abstract heading
\chapter*{Abstract}        % this creates the heading for the abstract page

\selectlanguage{english}
\vspace{-1cm}
\begin{adjustwidth}{1.5cm}{1.5cm}
	The main focus of this thesis is the signals of new physics signal in the
	LHC proton–proton collisions data collected by the ATLAS detector at 
	the Large Hadron Collider located at CERN.\\		
	The flavour-changing neutral-current (FCNC) processes are forbidden at tree-level and 
	highly suppressed at loop-level which is why they are very rare phenomena in the Standard Model of particle physics.
	However, this processes have an higher probability to occur in several models beyond the Standard Model where the 
	suppression could be relaxed and the loop diagrams mediated by new bosons that could contribute.\\	
	The FCNC top-quark decays $\Pqt\rightarrow\PZ\Pqc$ are searched for a signal of new physics using 
	\ttbar pair production events with one top quark decaying through the $\Pqt\rightarrow\PZ\Pqc$ channel 
	and the other through the dominant Standard Model mode $\Pqt\rightarrow\PW\Pqb$.
	The analyzed data were recorded at a center-of-mass energy of \SI{13}{\TeV} and correspond to the full Run-2 dataset 
	with an integrated luminosity of \SI{139}{\ifb}.\\
	The expected upper limit at 95\% CL is set at $\mathrm{BR (\Pqt\rightarrow\PZ\Pqc)=\SI{9.6e-5}{}}$, 
	improving the previous ATLAS results by a factor of 3.3.
	
	
\end{adjustwidth}

%\marginparsep = 20pt
%\hoffset = 10pt
%\restoregeometry
%\end{abstractlongs}


