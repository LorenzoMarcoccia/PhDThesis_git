%\newgeometry{top=2cm, left=2.7cm, right=3cm, bottom=1.45cm}
% Thesis Abstract -----------------------------------------------------
%\begin{abstractslong}   % uncommenting this line, gives a different abstract heading

\selectlanguage{english}
\thispagestyle{plain}
\begin{center}
	
	\Large
	\textbf{
		\newline
		\newline
		Search for flavour-changing neutral-current top quark decays to c-quark and Z boson using the ATLAS detector at the LHC}
		
	\vspace{0.5cm}
	\large{Lorenzo Marcoccia}
	
	\vspace{0.3cm}
	\large 
	University of Rome Tor Vergata, 2021
	
	\vspace{0.3cm}
	\normalsize 
	A thesis submitted in fulfilment of the requirements for the degree of\\
	 \textit{Doctor of Philosophy}
	
	\vspace{0.9cm}
	\large
	\textbf{Abstract}
	\vspace{0.3cm}
	
\end{center}
\begin{adjustwidth}{1.5cm}{1.5cm}
	The main focus of this thesis is the search for the $\Pqt\rightarrow\PZ\Pqc$ process in the 
	proton–proton collisions data collected by the ATLAS detector at 
	the Large Hadron Collider located at CERN.\\		
	The flavour-changing neutral-current (FCNC) processes are forbidden at tree-level and 
	highly suppressed at loop-level which is why they are very rare phenomena in the Standard Model of particle physics.
	However, this processes have an higher probability to occur in several theories beyond the Standard Model where the 
	suppression could be relaxed and the loop diagrams mediated by new bosons could contribute.\\	
	The FCNC top-quark decays $\mathrm{\Pqt\rightarrow\PZ\Pqc}$ are searched in 
	\ttbar pair events with one top quark decaying through the $\mathrm{\Pqt\rightarrow\PZ\Pqc}$ channel 
	and the other through the dominant Standard Model mode $\mathrm{\Pqt\rightarrow\PW\Pqb}$.
	The analysed data were recorded at a center-of-mass energy of \SI{13}{\TeV} and correspond to the full Run-2 dataset 
	with an integrated luminosity of \SI{139}{\ifb}.\\
	%	The expected upper limit at 95\% CL is set at $\mathrm{BR (\Pqt\rightarrow\PZ\Pqc)=\SI{9.6e-5}{}}$, 
	%	improving the previous ATLAS results by a factor of 3.3.
	The data are consistent with Standard Model background contributions, and, at 95\% confidence level the search sets observed (expected) upper limits of $\mathrm{\SI{11.8e-5}{}}$ ($\mathrm{\SI{9.5e-5}{}}$) on the $\mathrm{\Pqt\rightarrow\Pqc\PZ}$ branching ratio, constituting the most stringent limit to date and improving the previous\\ ATLAS results by a factor of 2 (2.5).
		
\end{adjustwidth}



