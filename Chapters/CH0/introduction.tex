
% this file is called up by thesis.tex
% content in this file will be fed into the main document

% ----------------------- introduction file header -----------------------
\chapter*{Introduction}
\label{chapter:introduction}
\markboth{INTRODUCTION}{Introduction}
% ----------------------- paths to graphics ------------------------

% the code below specifies where the figures are stored
\graphicspath{Chapters/CH0/figures}
% ----------------------------------------------------------------------
% ----------------------- introduction content -------------------------
% ----------------------------------------------------------------------
Developed between the late 1960s and the mid-1970s, the standard model (SM) of
particle physics represents our best understanding of physics phenomena at the most
fundamental scales. It provides a unified picture for all known elementary particles
and the way they interact via 3 of the 4 fundamental forces. \\
Over the decades, the SM has been tested extensively by a broad variety of
experiments. It is able to successfully explain almost all experimental results over a
wide energy range, at times with a precision unmatched in any other field of physics.
However, the SM is not the ultimate "theory of everything", there are some critical points that
arise both from theoretical considerations and experimental results that does not fit in the SM
and may indicate a presence of new physics Beyond the Standard Model (BSM).\\
Many are models of "new physics" that attempt to describe and explain these phenomena.
One way to test these models and to search for new physics is through the study of the top-quark, the heaviest elementary particle predicted by the Standard Model and discovered in 1995 by the CDF and D\O \ experiments at the Tevatron collider.\\ 
In the SM, top quark decays almost exclusively into bW, while flavour-changing neutral current (FCNC) decays, such as $\mathrm{\Pqt\rightarrow \PZ\Pqc}$, are forbidden at tree level. 
FCNC decays occur at one-loop level but are strongly suppressed by the GIM mechanism, with a suppression factor of 14 orders of magnitude relative to the dominant decay mode.
However, in the BSM models, the suppression could be relaxed and the loop diagrams mediated
by new bosons that could contribute, leading to couplings of many orders of magnitude higher
than those expected by the SM.
Therefore, any significant signal of top-quark FCNC decays will indicate the existence of new physics.
\vspace{\baselineskip}
\\In this thesis, a search for FCNC top-quark decays in a c-quark and a Z boson is presented taking into 
account also the FCNC process of production of a single top-quark in association with a Z boson.
The analysed data were recorded at a center-of-mass energy of \SI{13}{\TeV} by the ATLAS detector at 
the Large Hadron Collider located at CERN and correspond to the full Run-2 dataset 
with an integrated luminosity of \SI{139}{\ifb}.\\
\vspace{\baselineskip}
\\Additional work, concerning the ATLAS detector development, will also be presented in this thesis. 
The goal of this work was the development of a new model for the Resistive Plate Chambers (RPC) detectors in the Barrel Inner (BI) region and, using this model, to perform trigger efficiency studies that will help to plan the work for the Phase-II upgrade (2024-2025) that will lead LHC to High-Luminosity LHC (HL-LHC).
\vspace{\baselineskip}
\\My original contributions presented in this thesis include:
\begin{itemize}
	\item the construction of a model implementing a realistic digitization in the BI region, and some performance studies for the L0 barrel trigger;
	\item  the definition and optimisation of SR3tZc, and the investigation of the best \Pqc-tagging technique for this analysis;
	\item the determination of the backgrounds in SR3tZc, and the separation of signal from background events using a multivariate analysis;
	\item the full design of the analysis, including the fit strategy and the extraction of the signal limit. \\
\end{itemize}
\vspace{\baselineskip}
\noindent This thesis is organised as follows. The SM is introduced in  \Cref{chapter:SM}, including a
more detailed discussion on some BSM theories that have predictions on the FCNC top decay. 
In \Cref{chapter:ATLAS} the LHC accelerator and the ATLAS detector are presented. 
A brief description of the trigger system for HL-LHC is given in \Cref{chapter:upgrade} together with the studies performed for the Phase-II upgrade. 
In  \Cref{chapter:object_reco} the data modelling and the object reconstruction is presented.
\Cref{chapter:analysis} describes the analysis strategy developed for the search for FCNC couplings between top-quark and \PZ boson. \Cref{chapter:charm_tag} is devoted to the main analysis, with the description of Signal and Control Regions used for the statistical analysis described in \Cref{chapter:Statistical_analysis} in order to extract the observed upper limits at 95\% confidence level (CL) for the FCNC $\Pqt\rightarrow\PZ\Pqc$ process.
 
 