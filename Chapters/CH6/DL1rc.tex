
% this file is called up by thesis.tex
% content in this file will be fed into the main document

% ----------------------- introduction file header -----------------------
%\chapter{Search for the FCNC decay \pmb{$t\rightarrow c + Z$}}
\chapter{Search for the FCNC decay of top-quark in \Pqc-quark and Z boson}
\label{chapter:analysis}

% the code below specifies where the figures are stored
\graphicspath{Chapters/CH6/figures}

% ----------------------------------------------------------------------
% ----------------------- introduction content -------------------------
% ----------------------------------------------------------------------


\section{Event selections and reconstruction}
\label{sec:selection_dl1rc}

\subsection {Signal Region definition}
\subsubsection{SR1 selections}
The Signal Region called SR1 targets the \FCNCtZq in \ttbar decays.\\
In this channel, one of the \Pqt-quarks decays following the SM into a \PW boson and a
\Pqb-quark (called in the following \textit{SM top}), while the other
\Pqt-quark (called in the following \textit{FCNC top}) decays into a \PZ
boson and a \Pqc-quark. Only the
trileptonic channel is considered, i.e. when the \PZ boson from the
FCNC top decays leptonically and the \PW boson from the SM top
decays leptonically. Therefore the final state is characterised by
the presence of three leptons, a \Pqc-jet, a \Pqb-tagged jet and missing
transverse momentum from the escaping neutrino.  

\subsubsection{SR2 selections}
The Signal Region called SR2 targets the \FCNCtZq in single-top production.\\
In this channel, the
production of a single top-quark proceeds through an FCNC
interaction. The \Pqt-quark is produced in association with a \PZ
boson. Also in this case only the trileptonic channel is
considered. Therefore the final state is characterised by
the presence of three leptons, a \Pqb-tagged jet and missing
transverse momentum from the escaping neutrino. 

\clearpage
\section{Background estimation}
\label{sec:background}
A variety of background sources are considered and already discussed in \Cref{sec:background}.
In order to study the modelling of the main background samples, Control regions (CRs) are used in the fit to extract the normalisation of some relevant background sources.\\
Several CRs are defined:
\begin{itemize}
	\item \ttbar CR is designed to control the minor \ttbar
	background. As previously mentioned, this background enters the
	selection because of the presence of a mis-reconstructed lepton,
	i.e. a fake lepton. Since this background is small, the decision was
	taken to evaluate it using MC. Nevertheless, the normalisation is
	taken from data. 
	\item \ttZ CR is designed to control the \ttZ
	background. It is constructed by requiring the presence of more jets
	with respect to the jet multiplicity required in the SRs.
	\item Side-band CRs are designed to contain a mixture
	of the main background sources (\ttZ and diboson). They are
	constructed applying cuts on the top mass.
\end{itemize}

\subsection {Control Regions definition}
\label{sec:bkg:sel}
In the following, the event selection in the CRs is described.\\
\Cref{tab:bkg:crs} summarises the selection cuts in the various CRs.
\paragraph{\ttbar CR selections}
The \ttbar CR is defined by requiring that there is at least one pair
of opposite-sign but different-flavour leptons in the
event. Obviously, no cut on the invariant mass of the opposite-sign
leptons is applied. Concerning the jet multiplicity, there should be
at least one jet in the event, of which exactly one should be
\Pqb-tagged.

\paragraph{\ttZ CR selections}
\label{sec:bkg:ttz}
The \ttZ CR is defined by requiring the presence of at least four jets
and of exactly two \Pqb-tagged jets. Also the cut on the transverse
mass of the \PW boson is softened to \SI{30}{\GeV}. To be orthogonal 
with SR3, also a veto on the presence of a c-jet is required. 

\paragraph{Side-band CR1 selections}
\label{sec:bkg:sbcr1tzu}
The mass side-band CR1 is defined by requiring the presence of at least two jets
and of exactly one \Pqb-tagged jet.
The mass of the FCNC top-quark candidate, \mtopfcnc,
must be outside $2\sigma^{FCNC}$ from \mtopvalue, the mass of the
SM top-quark candidate, \mtopsm, must be also outside $2\sigma^{SM}$
from \mtopvalue.  In addition, a veto on the presence of a c-jet is required. 

\paragraph{Side-band CR2 selections}
\label{sec:bkg:sbcr2}
The mass side-band CR2 is defined by requiring the presence of exactly one or two jets
and of exactly one \Pqb-tagged jet.   The transverse mass is calculated using the momentum of the lepton associated with the $W$ boson, $\MET$ and azimuthal angle,$\phi$, between them: 
$\mT(\ell_{\PW},\Pgn) = \sqrt{2\pT^{\ell}\MET \left(1-\cos\Delta\phi\right)}$.
Events are required to have $\mT(\ell_{\PW},\Pgn) > \SI{40}{\GeV}$.\\
The mass of the SM top-quark candidate, \mtopsm, must be also outside $2\sigma^{SM}$
from \mtopvalue.\\
Event yields in the CRs are shown in \Cref{tab:bkg:yields:tzc}. 
As it can be noticed, the signal contribution in the various CRs is small.
\begin{table}[htbp]
	\small
	\caption{Event yields in the CRs for the \tZc coupling extraction. \TabErrStatSys} 
	\label{tab:bkg:yields:tzc}
	\centering
	% NB: add to main document: 
% \usepackage{siunitx} 
% \sisetup{separate-uncertainty,table-format=6.3(6)}  % hint: modify table-format to best fit your tables
\begin{tabular}{|p{0.10\textwidth}|>{\centering}p{0.08\textwidth}|>{\centering}p{0.08\textwidth}|>{\centering}p{0.08\textwidth}|>{\centering}p{0.09\textwidth}|>{\centering}p{0.09\textwidth}|>{\centering}p{0.09\textwidth}|>{\centering\arraybackslash}p{0.09\textwidth}|}
\toprule  
& {SR1} & {SR2} & {SR3} & {Side-band CR1} & {Side-band CR2} & {\ttZ CR} & {\ttbar CR}\\
\midrule 
    \ttZ+\tWZ   & 168 $\pm$ 22 & 33 $\pm$ 7 & 82 $\pm$ 11 & 88 $\pm$ 12 & 9.1 $\pm$ 2.1 & 164 $\pm$ 22 & 14.8 $\pm$ 1.9 \\ 
\ttW   & 5.8 $\pm$ 1.0 & 3.3 $\pm$ 0.6 & 2.04 $\pm$ 0.35 & 4.3 $\pm$ 0.7 & 2.5 $\pm$ 0.5 & 2.3 $\pm$ 0.5 & 27 $\pm$ 4 \\ 
\ttH   & 6.1 $\pm$ 1.0 & 0.88 $\pm$ 0.18 & 2.6 $\pm$ 0.4 & 2.3 $\pm$ 0.4 & 0.36 $\pm$ 0.07 & 5.4 $\pm$ 0.9 & 13.8 $\pm$ 2.1 \\ 
\VVLF   & 28 $\pm$ 17 & 35 $\pm$ 13 & 2.9 $\pm$ 2.0 & 25 $\pm$ 15 & 18 $\pm$ 7 & 0.20 $\pm$ 0.22 & 0.40 $\pm$ 0.21 \\ 
\VVHF   & 140 $\pm$ 100 & 160 $\pm$ 70 & 30 $\pm$ 22 & 130 $\pm$ 80 & 69 $\pm$ 28 & 13 $\pm$ 11 & 2.3 $\pm$ 1.4 \\ 
\tZq   & 47 $\pm$ 7 & 110 $\pm$ 18 & 13.8 $\pm$ 2.3 & 20 $\pm$ 4 & 9.9 $\pm$ 1.7 & 14.6 $\pm$ 2.9 & 0.90 $\pm$ 0.15 \\ 
\ttbar+Wt   & 21 $\pm$ 4 & 32 $\pm$ 11 & 3.7 $\pm$ 1.0 & 10 $\pm$ 4 & 9.1 $\pm$ 2.7 & 3.0 $\pm$ 1.2 & 102 $\pm$ 24 \\ 
Other fakes   & 10 $\pm$ 11 & 12 $\pm$ 12 & 1.4 $\pm$ 1.6 & 3 $\pm$ 5 & 10 $\pm$ 11 & 0.00 $\pm$ 0.06 & 0.12 $\pm$ 0.14 \\ 
Other   & 2.5 $\pm$ 1.5 & 3.8 $\pm$ 2.8 & 0.48 $\pm$ 0.25 & 2.2 $\pm$ 1.6 & 0.8 $\pm$ 2.6 & 1.1 $\pm$ 0.5 & 2.9 $\pm$ 1.5 \\ 
\midrule 
Total background  & 430 $\pm$ 110 & 390 $\pm$ 80 & 139 $\pm$ 25 & 280 $\pm$ 80 & 130 $\pm$ 32 & 203 $\pm$ 27 & 164 $\pm$ 25 \\ 
\midrule 
Data   & 433 & 443 & 143 & 331 & 169 & 197 & 156 \\ 
\midrule 
Data / Bkg.   & 1.00 $\pm$ 0.25 & 1.15 $\pm$ 0.24 & 1.03 $\pm$ 0.20 & 1.18 $\pm$ 0.35 & 1.30 $\pm$ 0.34 & 0.97 $\pm$ 0.14 & 0.95 $\pm$ 0.16 \\ 
\bottomrule 
\end{tabular} 

\end{table} 

\clearpage
\global\pdfpageattr\expandafter{\the\pdfpageattr/Rotate 90}
\begin{sidewaystable}[!htbp]
	\caption{
		Overview of the requirements applied for selecting events in the control regions.
	}%
	\label{tab:bkg:crs}
	\centering
	\footnotesize
	\begin{tabular}{c|c|c|c}
		\toprule
		\multicolumn{4}{c}{Common selections} \\
		\midrule
		\multicolumn{4}{c}{Exactly 3 leptons with $|\eta| < 2.5$ and $\pT(\ell_1)> \SI{27}{\GeV}$, $\pT(\ell_2)> \SI{15}{\GeV}$, $\pT(\ell_3)> \SI{15}{\GeV}$} \\
		\midrule
		\ttbar CR & \ttZ CR & Side-band CR1 & Side-band CR2 \\
		\midrule
		$\ge$ 1 \OS pair, no \OSSF  & $\ge$ 1 \OSSF pair & $\ge$ 1 \OSSF pair & $\ge$ 1 \OSSF pair \\
		& with $|\mll - \SI{91.2}{\GeV}| < \SI{15}{\GeV}$ & with $|\mll - \SI{91.2}{\GeV}| < \SI{15}{\GeV}$ & with $|\mll - \SI{91.2}{\GeV}| < \SI{15}{\GeV}$ \\
		-- & $\mT(\ell_{\PW},\Pgn) > \SI{30}{\GeV}$ & $\mT(\ell_{\PW},\Pgn) > \SI{40}{\GeV}$ & $\mT(\ell_{\PW},\Pgn) > \SI{40}{\GeV}$ \\
		$\ge$ 1 jet with $|\eta| < $ 2.5  & $\ge$ 4 jet with $|\eta| < $ 2.5 & $\ge$ 2 jet with $|\eta| < $ 2.5  & = 1 jet with $|\eta| < $ 2.5 \\
		= 1 \Pqb-jet  & = 2 \Pqb-jet  &  = 1 \Pqb-jet & = 1 \Pqb-jet \\
		-- & -- & = 0 c-jet & --\\ 
		-- & -- &  $|\mtopfcnc - \mtopvalue| > 2\sigma^{FCNC}$ & -- \\
		-- & -- & $|\mtopsm - \mtopvalue| > 2\sigma^{SM}$ & $|\mtopsm - \mtopvalue| > 2\sigma^{SM}$ \\
		\bottomrule
	\end{tabular}
\end{sidewaystable} 
\clearpage
\FloatBarrier
\subsubsection{\ttbar CR}
%
\global\pdfpageattr\expandafter{\the\pdfpageattr/Rotate 0} -------------------------------------------------------------------------------
\Cref{fig:sel:cr:ttbar:leps,fig:sel:cr:ttbar:jets} show the distributions 
of kinematic variables for events selected in the \ttbar CR region.

\begin{figure}[!htbp]
	\centering
	\begin{tabular}{cc}
		\includegraphics[width=.32\textwidth]{Chapters/CH6/figures/TTCR/lep1_eta} &
		\includegraphics[width=.32\textwidth]{Chapters/CH6/figures/TTCR/lep1_pt} \\
		\includegraphics[width=.32\textwidth]{Chapters/CH6/figures/TTCR/lep2_eta} &
		\includegraphics[width=.32\textwidth]{Chapters/CH6/figures/TTCR/lep2_pt} \\
		\includegraphics[width=.32\textwidth]{Chapters/CH6/figures/TTCR/lep3_eta} &
		\includegraphics[width=.32\textwidth]{Chapters/CH6/figures/TTCR/lep3_pt} \\
	\end{tabular}
	\caption{Pre-fit distributions of kinematic variables of leptons for events selected in the \ttbar CR.
		\ErrStatOnly
	}%
	\label{fig:sel:cr:ttbar:leps}
\end{figure}

\begin{figure}[htbp]
	\centering
	\begin{tabular}{cc}
		\includegraphics[width=.35\textwidth]{Chapters/CH6/figures/TTCR/nJets} &
		\includegraphics[width=.35\textwidth]{Chapters/CH6/figures/TTCR/jet_pt} \\
		\includegraphics[width=.35\textwidth]{Chapters/CH6/figures/TTCR/b_pt} & \includegraphics[width=.35\textwidth]{Chapters/CH6/figures/TTCR/b_eta} \\
	\end{tabular}
	\caption{Pre-fit distributions of kinematic variables of jets for events selected in the \ttbar CR.
		\ErrStatOnly
		\NoBDTGCut
		\Blinded
	}%
	\label{fig:sel:cr:ttbar:jets}
\end{figure}

\clearpage
\FloatBarrier
% -------------------------------------------------------------------------------
\subsubsection{\ttZ CR}
% -------------------------------------------------------------------------------
\Cref{fig:sel:cr:ttz:leps,fig:sel:cr:ttz:jets} show the distributions 
of kinematic variables for events selected in the \ttZ CR region.

\begin{figure}[!htbp]
	\centering
	\begin{tabular}{cc}
		\includegraphics[width=.32\textwidth]{Chapters/CH6/figures/TTZCR/lep1_eta} &
		\includegraphics[width=.32\textwidth]{Chapters/CH6/figures/TTZCR/lep1_pt} \\
		\includegraphics[width=.32\textwidth]{Chapters/CH6/figures/TTZCR/lep2_eta} &
		\includegraphics[width=.32\textwidth]{Chapters/CH6/figures/TTZCR/lep2_pt} \\
		\includegraphics[width=.32\textwidth]{Chapters/CH6/figures/TTZCR/lep3_eta} &
		\includegraphics[width=.32\textwidth]{Chapters/CH6/figures/TTZCR/lep3_pt} \\
	\end{tabular}
	\caption{Pre-fit distributions of kinematic variables of leptons for events selected in the \ttZ CR.
		\ErrStatOnly
	}%
	\label{fig:sel:cr:ttz:leps}
\end{figure}

\begin{figure}[htbp]
	\centering
	\begin{tabular}{cc}
		\includegraphics[width=.35\textwidth]{Chapters/CH6/figures/TTZCR/nJets} &
		\includegraphics[width=.35\textwidth]{Chapters/CH6/figures/TTZCR/jet_pt} \\
		\includegraphics[width=.35\textwidth]{Chapters/CH6/figures/TTZCR/b_pt} &
		\includegraphics[width=.35\textwidth]{Chapters/CH6/figures/TTZCR/b_eta} \\
	\end{tabular}
	\caption{Pre-fit distributions of kinematic variables of jets for events selected in the \ttZ CR.
		\ErrStatOnly
		\NoBDTGCut
		\Blinded
	}%
	\label{fig:sel:cr:ttz:jets}
\end{figure}

\clearpage
\FloatBarrier
% -----------------------------------------------------------------------------
\subsubsection{Side-band CR1}
% -------------------------------------------------------------------------------
\Cref{fig:sel:cr:sb1tzc:leps,fig:sel:cr:sb1tzc:jets} show the distributions 
of kinematic variables for events selected in the side-band CR1 region. 

\begin{figure}[!htbp]
	\centering
	\begin{tabular}{cc}
		\includegraphics[width=.32\textwidth]{Chapters/CH6/figures/SBCR1/lep1_eta} &
		\includegraphics[width=.32\textwidth]{Chapters/CH6/figures/SBCR1/lep1_pt} \\
		\includegraphics[width=.32\textwidth]{Chapters/CH6/figures/SBCR1/lep2_eta} &
		\includegraphics[width=.32\textwidth]{Chapters/CH6/figures/SBCR1/lep2_pt} \\
		\includegraphics[width=.32\textwidth]{Chapters/CH6/figures/SBCR1/lep3_eta} &
		\includegraphics[width=.32\textwidth]{Chapters/CH6/figures/SBCR1/lep3_pt} \\
	\end{tabular}
	\caption{Pre-fit distributions of kinematic variables of leptons for events selected in the side-band CR1 region.
		\ErrStatOnly
	}%
	\label{fig:sel:cr:sb1tzc:leps}
\end{figure}

\begin{figure}[htbp]
	\centering
	\begin{tabular}{cc}
		\includegraphics[width=.35\textwidth]{Chapters/CH6/figures/SBCR1/nJets}&
		\includegraphics[width=.35\textwidth]{Chapters/CH6/figures/SBCR1/jet_pt} \\
		\includegraphics[width=.35\textwidth]{Chapters/CH6/figures/SBCR1/b_pt} &
		\includegraphics[width=.35\textwidth]{Chapters/CH6/figures/SBCR1/b_eta}\\
	\end{tabular}
	\caption{Pre-fit distributions of kinematic variables of jets for events selected in the side-band CR1 region.
		\ErrStatOnly
		\NoBDTGCut
		\Blinded
	}%
	\label{fig:sel:cr:sb1tzc:jets}
\end{figure}

\clearpage
\FloatBarrier
% -------------------------------------------------------------------------------
\subsubsection{Side-band CR2}
% -------------------------------------------------------------------------------
\Cref{fig:sel:cr:sb2:leps,fig:sel:cr:sb2:jets} show the distributions 
of kinematic variables for events selected in the side-band CR2 region.

\begin{figure}[!htbp]
	\centering
	\begin{tabular}{cc}
		\includegraphics[width=.32\textwidth]{Chapters/CH6/figures/SBCR2/lep1_eta} &
		\includegraphics[width=.32\textwidth]{Chapters/CH6/figures/SBCR2/lep1_pt} \\
		\includegraphics[width=.32\textwidth]{Chapters/CH6/figures/SBCR2/lep2_eta} &
		\includegraphics[width=.32\textwidth]{Chapters/CH6/figures/SBCR2/lep2_pt} \\
		\includegraphics[width=.32\textwidth]{Chapters/CH6/figures/SBCR2/lep3_eta} &
		\includegraphics[width=.32\textwidth]{Chapters/CH6/figures/SBCR2/lep3_pt} \\
	\end{tabular}
	\caption{Pre-fit distributions of kinematic variables of leptons for events selected in the side-band CR2 region.
		\ErrStatOnly
	}%
	\label{fig:sel:cr:sb2:leps}
\end{figure}

\begin{figure}[htbp]
	\centering
	\begin{tabular}{cc}
		\includegraphics[width=.35\textwidth]{Chapters/CH6/figures/SBCR2/nJets} &
		\includegraphics[width=.35\textwidth]{Chapters/CH6/figures/SBCR2/jet_pt} \\
		\includegraphics[width=.35\textwidth]{Chapters/CH6/figures/SBCR2/b_pt} &
		\includegraphics[width=.35\textwidth]{Chapters/CH6/figures/SBCR2/b_eta} \\
	\end{tabular}
	\caption{Pre-fit distributions of kinematic variables of jets for events selected in the side-band CR2 region.
		\ErrStatOnly
		\NoBDTGCut
		\Blinded
	}%
	\label{fig:sel:cr:sb2:jets}
\end{figure}


%\clearpage
%\subsection {Fake composition}
%The origin of the three leptons from \ttbar events was studied, to
%understand where the fake leptons come from and eventually define an
%uncertainty to take into account the different sources in the signal
%and control regions. The origin of the three leptons from \ttbar
%events is shown in \cref{fig:bkg:fakes:ttbar:comp}. As it can be
%seen, apart from the leptons for which the mother is the top quark
%(\SI{66}{\%} of the leptons in each event), the fake leptons come from
%either photon conversions (around \SI{5}{\%}, depending on the region)
%or from \Pqb-hadrons (around \SI{25}{\%}, depending on the region). \\
%Some differences can be noticed for the photon conversion and
%\Pqb-hadron fractions in the signal regions with respect to the \ttbar
%control region where the \ttbar background is controlled. To take into
%account this differences, a systematic uncertainties is added. 
%
%\begin{figure}[htbp]
%	\centering
%	\includegraphics[width=1.\textwidth]{Chapters/CH5/figures/ttbar_leptons_origin}
%	\caption{Origin of the three leptons from \ttbar events in the
%		SR1\tZu, SR2\tZu and \ttbar CR. 
%	}%
%	\label{fig:bkg:fakes:ttbar:comp}
%\end{figure}
%
%\clearpage
%\FloatBarrier


\clearpage
\section{Separation of signal from background events}
\label{sec:separation_dl1r}
\clearpage
\subsection {GBDT discriminant definition}
\clearpage
\subsection {Input variables}
\clearpage


