
% this file is called up by thesis.tex
% content in this file will be fed into the main document

% ----------------------- introduction file header -----------------------
%\chapter{Search for FCNC $t\rightarrow Zc$ using a charm-tagger}
%\chapter{Search for FCNC $t\rightarrow Zc$}
\chapter{Search for FCNC \texorpdfstring{\MakeLowercase t}{t}$\rightarrow$ Z\texorpdfstring{\MakeLowercase c}{c}}
\label{chapter:charm_tag}
This chapter presents an important fraction of my work. It is dedicated to the search for FCNC $t\rightarrow Zc$ using the SMT technique, the charm-tagger \DLrc and the comparison between these two techniques.
The design and optimization of the multivariate analysis is part of my work as well, and it is presented in this chapter.

% ----------------------- paths to graphics ------------------------

\graphicspath{Chapters/CH6/figures}

\section{Event selection and reconstruction}
\label{sec:selection}
This section describes the event selection for the search for FCNC \tZc decay.
One of the \Pqt-quarks decays following the SM into a \PW boson and a
\Pqb-quark (called in the following \textit{SM top}), while the other
\Pqt-quark (called in the following \textit{FCNC top}) decays into a \PZ
boson and a \Pqc-quark that subsequentially decays semi-leptonically.
This semi-leptonic decay is tagged using the Soft Muon Tagging (SMT) technique. \\
Only the trileptonic channel is considered, i.e. the \PZ boson from the
FCNC top decays leptonically and the \PW boson from the SM top
decays leptonically.
Therefore the final state is characterised by the presence of three leptons, an
SMT-jet, a \Pqb-tagged jet and missing transverse momentum from the
escaping neutrino. \\
The final states where either the \PZ or the \PW bosons decay
hadronically are not considered because of the higher backgrounds.\\
The pre-selection criteria, common to all the Signal Regions used in this work, are the following:
\begin{itemize}
	\item    Exactly three isolated leptons (electrons or muons) are required. 
				These leptons must satisfy the requirements described in
				\Cref{sec:object:el,sec:object:mu}. 
				At least one lepton must have $\pt > \SI{27}{\GeV}$ to be above the trigger threshold.
				The other two leptons must have $\pt > \SI{15}{\GeV}$. 
				Events with a fourth lepton with $\pt > \SI{15}{\GeV}$ are vetoed. 
	\item   There should be at least one opposite-sign same-flavour lepton pair
				(OSSF) with an invariant mass in the range 
				$|\mll - \SI{91.2}{\GeV}| < \SI{15}{\GeV}$. 
				These two leptons are considered to be the ones coming from the \PZ boson.\\ 	
				If more than one lepton pair satisfies these selections, the pair
				with the invariant mass closest to the mass of the \PZ boson is
				considered to be from the \PZ decay. 
\end{itemize}	

\subsection {Top quarks reconstruction}
\label{sec:sel:topmassrec}
The next step is the reconstruction of the two top quarks.
The signal event has at least two jets with one of them being \Pqb-tagged, two top quark (FCNC and SM tops) candidates are reconstructed under the FCNC \ttbar decay signal hypothesis.
The kinematics of the top-quark candidates can be reconstructed from
the corresponding decay particles.\\
The reconstructed \PZ boson is assumed to come from the FCNC top decay ($t\to cZ$),
while the \Pqb-tagged jet from SM top decay ($t\to bW$).\\
In order to reconstruct both top quarks, we need to associate a reconstructed jet to the \Pqc-quark from the FCNC top decay, and to reconstruct
the \PW boson from the SM top decay. This can be done by assuming that the lepton not used to reconstruct the \PZ boson is the one coming from the \PW boson decay,
the missing transverse momentum is the transverse momentum of the neutrino from \PW boson decay and determining the longitudinal component of the neutrino momentum ($p^{\nu}_z$)
using the minimisation of the following expression for each jet combination:

\begin{equation}
\chi^2_{\ttbar}  =  \frac{\left(m^{\mathrm{reco}}_{j_a\ell\ell}-m_{t_{\mathrm{FCNC}}}\right)^2}{\sigma_{t_{\mathrm{FCNC}}}^2}+
\frac{\left(m^{\mathrm{reco}}_{j_b\ell_W\nu}-m_{t_{\mathrm{SM}}}\right)^2}{\sigma_{t_{\mathrm{SM}}}^2}
+ \frac{\left(m^{\mathrm{reco}}_{\ell_W\nu}-m_W\right)^2}{\sigma_W^2},
\label{eq:chi2tt}
\end{equation}
where $m^{\mathrm{reco}}_{j_a\ell\ell}$, 
$m^{\mathrm{reco}}_{j_b\ell_W\nu}$, and $m^{\mathrm{reco}}_{\ell_W\nu}$ are 
the reconstructed masses of the $cZ$, $bW$, and $\ell_W\nu$ systems, respectively.
For each jet combination, where any jet can be assigned to $j_a$, while $j_b$ must correspond to a \Pqb-tagged jet, the $\chi^2_{\ttbar}$ minimisation gives the most probable value for $p^{\nu}_z$. From all combinations, the one with the minimum $\chi^2$ is chosen.
%This procedure assigns the SMT-jet to the \Pqc-quark from FCNC top decay or to the \Pqb-quark 
%from SM top decay (depending on the number of \Pqb-tagged jet) and determines the $p^{\nu}_z$ value, completing all ingredients to reconstruct four-momenta of the 
%top-quark candidates.
Since a semi-leptonic decay can be originated from a C-hadron decay but also from a B-hadrons decay (see \Cref{sec:smt_compositions}), the SMT-jet can be associated to the FCNC top decay or to the SM top decay, depending on the minimum $\chi^2$. When the \DLrc charm tagger is used (see \Cref{sec:other_selection}), the c-tagged jet is assigned to $j_a$.
\\ In~\Cref{eq:chi2tt}, the central value for the masses and the widths of the top quarks and \PW boson are taken from reconstructed simulated FCNC \ttbar decay signal events. This is done by matching the true \Pqc- and \Pqb-quarks in the simulated events to the reconstructed ones, setting the longitudinal momentum of the neutrino to the $p_z$ of the true simulated neutrino and then performing Bukin fits\footnote{These fits use a piecewise function with  a Gaussian function in the centre and two 		
	asymmetric tails. Five parameters determine the overall normalization, the peak position, the width of the core, the asymmetry, the size of the lower tail, and the size of the higher tail. Of these parameters, only the peak position and the width enter the $\chi^2$.}~\cite{Bukin} 
to the masses of the reconstructed top quarks and \PW boson (more details are in Appendix~\ref{appendix:mass_resolution}). 
The extracted values are: 
\begin{itemize}
	\item $m_{t_{\mathrm{FCNC}}}=171.2$~\GeV,  $\sigma_{t_{\mathrm{FCNC}}}=11.4$~\GeV;
	\item $m_{t_{\mathrm{SM}}}=168.0$~\GeV,  $\sigma_{t_{\mathrm{SM}}}=23.9$~\GeV;
	\item $m_W=82.6$~\GeV,  $\sigma_W=16.6$~\GeV.
\end{itemize}	
The SM top quark candidate is reconstructed under the FCNC single-top quark production hypothesis in the events having one or two jets with exactly one being $b$-tagged, which is assumed to come from the top-quark decay ($t\to bW$). 
The most probable value for the longitudinal component of the neutrino momentum for the FCNC single-top quark production is determined using the minimisation of the following expression:
\begin{equation}
\chi^2_{\tZ}  = 
\frac{\left(m^{\mathrm{reco}}_{b\ell_W\nu}-m_{t_{\mathrm{SM}}}\right)^2}{\sigma_{t_{\mathrm{SM}}}^2}
+ \frac{\left(m^{\mathrm{reco}}_{\ell_W\nu}-m_W\right)^2}{\sigma_W^2},
\label{eq:chi2tZ}
\end{equation}
where $m^{\mathrm{reco}}_{j_b\ell_W\nu}$ and $m^{\mathrm{reco}}_{\ell_W\nu}$ are 
the reconstructed masses of the $bW$ and $\ell_W\nu$ systems, respectively.
In~\Cref{eq:chi2tZ}, the central value for the masses and the widths
of the top quark and $W$ boson are the same as in~\Cref{eq:chi2tt}, therefore, in the events with two jets,
the four-momentum of SM top-quark candidate reconstructed under the FCNC single-top quark
production signal hypothesis is the same as the one reconstructed under the FCNC \ttbar decay signal hypothesis. 
Moreover, the first term (the FCNC top) in~\Cref{eq:chi2tt} is a constant term in the minimisation of $\chi^{2}$, therefore it does not affect the extraction of the longitudinal component of the neutrino momentum.


\subsection{Main sources of background}
\label{sec:background}
A variety of background sources are considered.
These include SM processes with three leptons in the final state as the FCNC \ttbar (Zc)
process (such as VV or the associated production of \ttbar with
a \PZ boson), as well as events in which at least one of the leptons
in the final state is \textit{fake} (either a jet misidentified as a
lepton or a non-prompt lepton).
The estimation of the various sources of background relies on MC
simulations while for the
\ttbar fake-lepton background the shapes are taken from MC events but the
normalisation is extracted from data. \\
The \ttZ process enters the event selection because of the presence in the
final state of a SM top quark and of a \PZ boson. The only difference
w.r.t the signal topology, for the semi-leptonic \ttbar decay, is the
presence of additional jets in the event. \\
The diboson background (mainly \PW\PZ and \PZ\PZ) enters the selection because of the presence
of three leptons and of additional jets emitted, that can come from heavy
quarks. 
The diboson background is split into \VVHF (heavy flavour) and \VVLF (light flavour) based on the type of jets associated:
if one of the associated jets originated from a \Pqb-quark or a \Pqc-quark then it is considered as \VVHF, otherwise it is considered as \VVLF. % chktex 13
The jet type is determined in simulations using the \texttt{jet\_truthflav} variable.
This variable, provided by the flavour tagging group, defines a cone of $\Delta R < 0.3$ associated with each jet.
If a \Pqb-hadron with \pT > \SI{5}{\GeV} is found within this cone the jet is identified as a \bjet.
If no \Pqb-hadrons are found, the algorithm searches for \Pqc-hadrons, then \Pgt leptons.
If none of these identifiers are found the jet is labelled as a light jet.\\
In the following sections, some other backgrounds are grouped as follows:
\begin{itemize}
	\item \ttZ+\tWZ;
	\item \ttW+\ttH;
	\item \ttbar+\Wt, backgrounds with fake leptons;
	\item \textit{Other fakes} which contains backgrounds with fake leptons as well (\Zjets, \VV~(2l), \ttZ~(2l)) but categorized in a different group as explained in \Cref{sec:stat:strategy};
	\item \textit{Other} which contains minor backgrounds (\ttt, \VVV,  \ttWW etc.).
\end {itemize}

\section {Signal Region with SMT requirement}
\label{sec:sel:sr3}
The Signal region considered in this chapter is called SR3\tZc and it has the following requirements in addition to the cuts described previously:
\begin{itemize}
	\item At least two jets satisfying the requirements described in \Cref{sec:object:jet}. 
	\item Zero, one or two \Pqb-jets satisfying the requirements in \Cref{sec:object:bjet}. 
	\item The selected soft muon must be opposite-sign with the lepton coming from \PW boson also satisfying the requirements described in \Cref{sec:object:soft_muons} .
	\item At least one SMT jet described in \Cref{sec:object:smt}. For each event that contains at least one 
	soft muon, the SMT-jet must be assigned to the FCNC top or the SM top, depending on the $\chi^2$ 
	in~\Cref{eq:chi2tt}.	 
	\item No requirements on the masses of both the FCNC and the SM top-quark candidates are applied. 
\end{itemize}
These selections are summarized in Table~\ref{tab:sel:sr3_smt}.\\
The event yields for each \Pqb-jet multiplicity and the total event yields are shown in Table~\ref{tab:yields:sr3}. 
Even though the selection with exactly one \Pqb-jet is the purest, events containing zero and two \Pqb-jets are also considered in order to have the largest possible signal acceptance and then work on the separation of signal from background, as described in \Cref{sec:separation}.\\
~\Cref{fig:sr3_kin_lep,fig:sr3_kin_jet} show the distributions of some kinematic variables for events selected in the SR3 region for the \tZc coupling extraction selection (SR3\tZc). 
As it can be noticed, the main background sources are \ttZ and \VVHF. In the next section these
two backgrounds will be investigated more in detail exploiting the information carried by the soft muon
decay chain.

\begin{table}[!h]
	\centering
	\small
	\begin{tabular}{c}
		\toprule
		SR3\tZc using SMT\\
		\midrule
		Exactly 3 leptons with $|\eta| < 2.5$ and $\pT(\ell_1)> \SI{27}{\GeV}$, $\pT(\ell_2)> \SI{15}{\GeV}$, $\pT(\ell_3)> \SI{15}{\GeV}$\\
		$\ge$ 1 \OSSF pair, with $|\mll - \SI{91.2}{\GeV}| < \SI{15}{\GeV}$\\
		$\ge$ 2 jets with $|\eta| < $ 2.5\\
		$\le$ 2 \Pqb-jets\\
		OS($\mu^{soft}$,$\ell_W$)\\
		$\ge$ 1 SMT jet \\
		\bottomrule
	\end{tabular}
	\caption{Overview of the requirements applied to select events in the Signal Region with SMT.}
	\label{tab:sel:sr3_smt}
\end{table} 

\begin{table}[!h]
	\centering
	\footnotesize
	\begin{tabular}{|l|c|c|c|c|}
    \hline
    \multirow{2}{*}{\textbf{Sample}}      & \multicolumn{3}{|c|}{\textbf{Number of $b$-jets}} & \multirow{2}{*}{\textbf{Total yield}}\\
    \cline{2-4}
    & \textbf{=0}     & \textbf{=1}     & \textbf{=2}    &\\                    
    \hline   
    ZZ+LF &                 15.40 $\pm$ 0.24  & 0.41 $\pm$ 0.04   & 0.00 $\pm$ 0.00   &  15.81 $\pm$ 0.25  \\
    ZZ+HF &                 4.64  $\pm$ 0.12  & 5.15 $\pm$ 0.13   & 0.83 $\pm$ 0.03   &  10.63 $\pm$ 0.18  \\
    WZ+LF &                 20.13 $\pm$ 0.36  & 0.45 $\pm$ 0.06   & 0.01 $\pm$ 0.01   &  20.59 $\pm$ 0.37  \\
    WZ+HF &                 40.24 $\pm$ 0.51  & 21.92$\pm$ 0.38   & 3.10 $\pm$ 0.12   &  65.27 $\pm$ 0.65  \\
    VV (2l) &               0.05 $\pm$ 0.08   & 0.09 $\pm$ 0.04   & 0.00 $\pm$ 0.00   &  0.15 $\pm$ 0.09   \\
    tWZ &                   1.87 $\pm$ 0.19   & 7.80 $\pm$ 0.40   & 3.59 $\pm$ 0.26   &  13.26 $\pm$ 0.51  \\
    $t\bar{t}W$ &           0.23 $\pm$ 0.04   & 1.28 $\pm$ 0.10   & 1.04 $\pm$ 0.09   &  2.55 $\pm$ 0.14   \\
    $t\bar{t}Z$ (2l) &      0.00 $\pm$ 0.00   & 0.02 $\pm$ 0.02   & 0.00 $\pm$ 0.00   &  0.02 $\pm$ 0.02   \\
    $t\bar{t}Z$ &           7.69 $\pm$ 0.20  & 37.81 $\pm$ 0.45  & 38.35 $\pm$ 0.46   &  83.85 $\pm$ 0.67  \\
    Wt &                    0 $\pm$ 0.00      & 0.27 $\pm$ 0.19   & 0.00 $\pm$ 0.00   &  0.27 $\pm$ 0.19   \\
    tZ &                    8 $\pm$ 0.15      & 14.85 $\pm$ 0.30  & 5.60 $\pm$ 0.16   &  23.63 $\pm$ 0.37  \\
    $t\bar{t}$ &            0.91 $\pm$ 0.19   & 3.97 $\pm$ 0.39   & 1.29 $\pm$ 0.22   &  6.17 $\pm$ 0.48   \\
    Z+jets &                2.85 $\pm$ 0.80   & 1.07 $\pm$ 0.65   & 0.20 $\pm$ 0.20   &  4.12 $\pm$ 1.05   \\
    4 tops &                0.01 $\pm$ 0.00   & 0.05 $\pm$ 0.01   & 0.15 $\pm$ 0.01   &  0.20 $\pm$ 0.01   \\
    3 tops &                0.00 $\pm$ 0.00   & 0.01 $\pm$ 0.00   & 0.02 $\pm$ 0.00   &  0.03 $\pm$ 0.00   \\
    VVV &                   37 $\pm$ 0.02     & 0.11 $\pm$ 0.01   & 0.01 $\pm$ 0.00   &  0.49 $\pm$ 0.03   \\
    VH &                    5 $\pm$ 0.64      & 0.80 $\pm$ 0.57   & 0.00 $\pm$ 0.00   &  1.76 $\pm$ 0.86   \\
    $t\bar{t}H$ &           0.24 $\pm$ 0.02   & 1.37 $\pm$ 0.04   & 1.51 $\pm$ 0.04   &  3.12 $\pm$ 0.05   \\
    $t\bar{t}WW$ &          0.01 $\pm$ 0.01   & 0.09 $\pm$ 0.03   & 0.13 $\pm$ 0.03   &  0.23 $\pm$ 0.05   \\
    \hline                                                                            
    Total bkg. &            98.79 $\pm$ 1.29  & 97.53 $\pm$ 1.25  & 55.82 $\pm$ 0.64  &  252.14 $\pm$ 1.91  \\
    \hline                                                                            
    FCNC $t\bar{t}$(cZ) &        3.44 $\pm$ 0.02   & 10.65 $\pm$ 0.04  & 1.40 $\pm$ 0.02   &  15.49 $\pm$ 0.05  \\
    FCNC (c)tZ &            0.31 $\pm$ 0.02   & 1.34 $\pm$ 0.03   & 0.10 $\pm$ 0.01   &  1.74 $\pm$ 0.04  \\
    \hline                                                                                     
%		$S/\sqrt{B}$ (ctZ) &   0.38 $\pm$ 0.01           & 1.21 $\pm$ 0.02      & 0.20 $\pm$ 0.00  & 1.09 $\pm$ 0.01 \\
%		\hline		                                                                        
\end{tabular}

	\caption{Event yields for each \Pqb-jet multiplicity and total event yields for the SR3\tZc selection.}
	\label{tab:yields:sr3}
\end{table}    

\begin{figure}[!htbp]
	\centering
	\begin{tabular}{cc}
		\includegraphics[width=.45\textwidth]{Chapters/CH6/figures/SR3_UsingSMT/lep1_eta} &
		\includegraphics[width=.45\textwidth]{Chapters/CH6/figures/SR3_UsingSMT/lep1_pt} \\
		\includegraphics[width=.45\textwidth]{Chapters/CH6/figures/SR3_UsingSMT/lep2_eta} &
		\includegraphics[width=.45\textwidth]{Chapters/CH6/figures/SR3_UsingSMT/lep2_pt} \\
		\includegraphics[width=.45\textwidth]{Chapters/CH6/figures/SR3_UsingSMT/lep3_eta} &
		\includegraphics[width=.45\textwidth]{Chapters/CH6/figures/SR3_UsingSMT/lep3_pt} \\
	\end{tabular}
	\caption{Pre-fit distributions of kinematic variables of leptons for events selected in the SR3\tZc region.  Number of signal events are normalised to the current observed branching ratio limits and scaled by factor 5. 
		\ErrStatOnly
		\Blinded
	}%
	\label{fig:sr3_kin_lep}
\end{figure}

\begin{figure}[htbp]
	\centering
	\begin{tabular}{cc}
		\includegraphics[width=.45\textwidth]{Chapters/CH6/figures/SR3_UsingSMT/nJets} &
	    \includegraphics[width=.45\textwidth]{Chapters/CH6/figures/SR3_UsingSMT/nbJets}\\
		\includegraphics[width=.45\textwidth]{Chapters/CH6/figures/SR3_UsingSMT/q_pt} &
		\includegraphics[width=.45\textwidth]{Chapters/CH6/figures/SR3_UsingSMT/b_pt} \\
		\includegraphics[width=.45\textwidth]{Chapters/CH6/figures/SR3_UsingSMT/softmu_dRmin} &
		\includegraphics[width=.45\textwidth]{Chapters/CH6/figures/SR3_UsingSMT/SMTjet_Pt} \\
	\end{tabular}
	\caption{Pre-fit distributions of kinematic variables of jets for events selected in the SR3\tZc region. Number of signal events are normalised to the current observed branching ratio limits and scaled by factor 5. 
		\ErrStatOnly
		\Blinded
	}%
	\label{fig:sr3_kin_jet}
\end{figure}
\newpage
\subsection {Reconstruction of the soft muon decay chain}
\label{sec:smt_compositions}
In \ttbar events, a soft muon in jets can be originated by various sources. In MC simulation, truth information can be used to determine the origin of the soft muon and the truth flavour of the SMT-jet that contains the soft muon. Therefore it is possible to reconstruct the chain of ancestors which in the end produces the soft muon. 
Four categories of events can be identified:
\begin{itemize}
	\item muons originating from the decay chain of a b-quark produced by a $t\rightarrow bW$ decay if the hadron and the b-quark are spatially matched within $\Delta$ R <0.4. Events with muons from $b\rightarrow \mu$, $b\rightarrow c\rightarrow \mu$ and $b\rightarrow \tau \rightarrow \mu$, are included in this category;
	\item muons originating from the decay chain of a c-quark produced by a $t\rightarrow Zc$ decay if the hadron and the c-quark are spatially matched within $\Delta$ R <0.4. Events with muons from c$\rightarrow \mu$ and $c\rightarrow \tau \rightarrow \mu$, are included in this category;
	\item muons which are either produced by light hadrons coming from a top-quark decay\\ ($t\rightarrow Wb$ or $t\rightarrow Zc$) or muons coming from the decay in flight of light hadrons, mostly pions and kaons. These muons can be also categorised as 'fake-SMT';
	\item muons that are effectively prompt leptons from a \PW or \PZ boson decay, failing the prompt lepton selection cuts, being close to a jet and therefore entering the soft muon selection criteria, referred to as prompt$\rightarrow \mu$.
\end{itemize}
According to the categories described above, Table~\ref{tab:sig_comp} shows the composition for the FCNC $t\bar{t}$(cZ) signal. Soft muons mostly come from B-hadrons ($\sim$ 60\%)  and C-hadrons ($\sim$ 40\%) decays.\\
Table~\ref{tab:ttZ_comp} and Table~\ref{tab:VV_comp} show the composition for the main backgrounds, \ttZ and \VVHF respectively. 
For \ttZ the main contributions come from B-hadrons ($\sim$ 80\%) as expected by the \ttZ topology.
For \VVHF the main contributions come from C-hadrons ($\sim$ 60\%), mostly  \WZ~\texttt{+$c\bar{c}$}.

\begin{table}[!h]
	\centering        
	\footnotesize  
	\begin{tabular}{|l|c|}          
	\hline          
	\multicolumn{2}{|l|}{\textbf{FCNC $t\bar{t}$(cZ)}}    \\        
	\hline          
	\multicolumn{2}{|l|}{\textbf{Total number of events = 15.49}}    \\
	\hline
	\textbf{Chain}        									 & \textbf{Fractions $[\%]$} \\                          
	\hline          
	b$\rightarrow \mu$                 					&   15.60  \\          
	b$\rightarrow$c$\rightarrow \mu$     	&    10.16   \\          
	b$\rightarrow \tau \rightarrow \mu$  	&    0.80 \\          
	c$\rightarrow \mu$                 				 &    57.46\\          
	c$\rightarrow \tau \rightarrow \mu$  	&     0.70 \\          
	light$\rightarrow \mu$              			&    10.80  \\          
	prompt$\rightarrow \mu$                	 	&   4.48 \\            
	\hline    
\end{tabular}    

	\caption{Reconstructed chain of ancestors that produces the soft muon for the FCNC $t\bar{t}$(cZ) signal.} %The soft muon is mostly coming from B-hadrons ($\sim$ 60\%)  and C-hadrons ($\sim$ 40\%). }
	\label{tab:sig_comp}
\end{table}    
\begin{table}[!h]
	\centering     
	\footnotesize     
	\begin{tabular}{|l|c|}          
	\hline          
	\multicolumn{2}{|l|}{\textbf{$t\bar t$Z}}    \\        
	\hline          
	\multicolumn{2}{|l|}{\textbf{Total number of events = 83.85}}    \\
	\hline
	\textbf{Chain}        									 & \textbf{Fractions $[\%]$} \\                          
	\hline          
	b$\rightarrow \mu$                 					&   42.24  \\          
	b$\rightarrow$c$\rightarrow \mu$     	&    31.82    \\          
	b$\rightarrow \tau \rightarrow \mu$  	&    3.01 \\          
	c$\rightarrow \mu$                 				 &    12.95\\          
	c$\rightarrow \tau \rightarrow \mu$  	&     0.14\\          
	light$\rightarrow \mu$              			&    7.32 \\          
	prompt$\rightarrow \mu$                	 	&   2.52\\            
	\hline    
\end{tabular}    

	\caption{Reconstructed chain of ancestors that produces the soft muon for the \ttZ background.}
	\label{tab:ttZ_comp}
\end{table}    
\begin{table}[!h]
	\centering    
	\footnotesize      
	\begin{tabular}{|l|c|}          
	\hline          
	\multicolumn{2}{|l|}{\textbf{\VVHF}}    \\        
	\hline          
	\multicolumn{2}{|l|}{\textbf{Total number of events = 75.90}}    \\
	\hline
	\textbf{Chain}        									 & \textbf{Fractions $[\%]$} \\                          
	\hline          
	b$\rightarrow \mu$                 					&   6.60  \\          
	b$\rightarrow$c$\rightarrow \mu$     	&    9.23    \\          
	b$\rightarrow \tau \rightarrow \mu$  	&    0.80 \\          
	c$\rightarrow \mu$                 				 &    57.46\\          
	c$\rightarrow \tau \rightarrow \mu$  	&     0.60\\          
	light$\rightarrow \mu$              			&    10.83 \\          
	prompt$\rightarrow \mu$                	 	&  4.48\\            
	\hline    
\end{tabular}    

	\caption{Reconstructed chain of ancestors that produces the soft muon for the \VVHF background.}
	\label{tab:VV_comp}
\end{table}    

%\clearpage
%\subsection {Fake composition}
%The origin of the three leptons from \ttbar events was studied, to
%understand where the fake leptons come from and eventually define an
%uncertainty to take into account the different sources in the signal
%and control regions. The origin of the three leptons from \ttbar
%events is shown in \cref{fig:bkg:fakes:ttbar:comp}. As it can be
%seen, apart from the leptons for which the mother is the top quark
%(\SI{66}{\%} of the leptons in each event), the fake leptons come from
%either photon conversions (around \SI{5}{\%}, depending on the region)
%or from \Pqb-hadrons (around \SI{25}{\%}, depending on the region). \\
%Some differences can be noticed for the photon conversion and
%\Pqb-hadron fractions in the signal regions with respect to the \ttbar
%control region where the \ttbar background is controlled. To take into
%account this differences, a systematic uncertainties is added. 
%
%\begin{figure}[htbp]
%	\centering
%	\includegraphics[width=1.\textwidth]{Chapters/CH6/figures/ttbar_leptons_origin}
%	\caption{Origin of the three leptons from \ttbar events in the
%		SR1\tZu, SR2\tZu and \ttbar CR. 
%	}%
%	\label{fig:bkg:fakes:ttbar:comp}
%\end{figure}
%
%\clearpage
%\FloatBarrier


\section{Separation of signal from background events}
\label{sec:separation}
Given the selection requirements in Table~\ref{tab:sel:sr3_smt}, a multivariate analysis (MVA) technique is used to have a better separation of the signal from the background and to increase the value of \ssplusb.\\ 
The Gradient Boosted Decision Trees (GBDT) method with TMVA software package~\cite{BDTG,TMVA} is exploited in this study. 
The output of this algorithm (called GBDT score) is correlated with \ssplusb and it is in the range between -1 and 1.
The most signal-like events have scores near 1 while the most background-like events have scores near -1. The GBDTs are trained separately in each signal regions as described below.
The SR3\tZc is defined targeting the FCNC \tZc coupling in \ttbar decay events
using the soft muon tagging, therefore the MVA discriminant, \Dthree, is built using a GBDT trained with FCNC \tZc \ttbar decay events against all backgrounds.

\subsection {Input variables}
\label{sec:input_var}
A set of variables as the GBDT input is used to train and test the GBDT method on the events in SR3\tZc. Those variables are listed in \Cref{tab:D3input}, ordered by the separation value, defined by, as in~\cite{TMVA}:
\begin{equation*}
\langle s^{2}\rangle = \frac{1}{2}\int \frac{[p_{s}(y)-p_{b}(y)]^{2}}{p_{s}(y)+p_{b}(y)}dy
\end{equation*}
where $p_{s}(y)$ and $p_{b}(y)$ are the signal and background PDFs of the classifier $y$. 
The separation is 0 (1) for identical (non-overlapping) signal and background shapes.\\
The set of input variables presented in this section has been constructed and optimized based on separation values, correlations and impact on the BDT performance. The details of the optimization procedure are documented in Appendix~\ref{app:BDT}.  
The distributions of input variables in the Signal Region are presented in~\Cref{fig:separation:SR3}.

\begin{table}[!htbp]
	\small
	\centering
	\begin{tabular}{ccc}
		\toprule
		Variable & $\langle s^{2}\rangle$  & Definition \\
		\midrule
		$m_{b\ell\nu}$  &  0.1717  &  SM top-quark candidate mass  \\
		$N\,b\,jets$  &  0.08218  &  Number of b-jets tagged with DL1r  \\
		$m_{q\ell\ell}$  &  0.07019  &  FCNC top-quark candidate mass  \\
		$\frac{\mu^{soft} ID p_{T}}{SMT\,jet\,Sum p_{T} Trk}$  &  0.03357  &  Ratio between the soft muon ID pT and pT sum of tracks  \\
		$\Delta R(\ell,Z)$  &  0.03141  &  $\Delta R$ between $W$ boson lepton and $Z$ boson candidates  \\
		$\Delta R(t_{\text{SM}},t_{\text{FCNC}})$  &  0.02508  &  $\Delta R$ between SM and FCNC top-quark candidates  \\
		$\Delta R(\mu^{soft},Z)$  &  0.006596  &  $\Delta R$ between soft muon and $Z$ boson candidates  \\
		\bottomrule
	\end{tabular}
	\caption{
	Set of variables used in the training of the GBDT in SR3\tZc to build the \Dthree discriminant. Variables are ordered by the separation $\langle s^{2}\rangle$ value. }
	\label{tab:D3input}
\end{table}

\begin{figure}[!htbp]
	\centering
	\begin{tabular}{cc}
		\includegraphics[width=.45\textwidth]{Chapters/CH6/figures/SR3_UsingSMT/tFCNC} &
		\includegraphics[width=.45\textwidth]{Chapters/CH6/figures/SR3_UsingSMT/tSM} \\
		\includegraphics[width=.45\textwidth]{Chapters/CH6/figures/SR3_UsingSMT/ttbar_dR} &
		\includegraphics[width=.45\textwidth]{Chapters/CH6/figures/SR3_UsingSMT/softmu_pt_id_SumPtTrkPt500} \\
		\includegraphics[width=.45\textwidth]{Chapters/CH6/figures/SR3_UsingSMT/ttbar_dR} &
		\includegraphics[width=.45\textwidth]{Chapters/CH6/figures/SR3_UsingSMT/softmuZ_dR} \\
	\end{tabular}
	\caption{Pre-fit distributions of the input variables used in the training of the GBDT in SR3\tZc to build the \Dthree discriminant. Number of signal events are normalised to the current observed branching ratio limits and scaled by factor 5.  
		\ErrStatOnly
		\Blinded
	}%
	\label{fig:separation:SR3}
\end{figure}

\clearpage

\subsection {GBDT training and evaluation}
In order to train the GBDT algorithm and have a reliable model with a good performance, it is better to use as much statistics as possible from the available signal and background MC samples.
On the other hand, to check the performance and validate the model, 
the trained GBDT model must be applied on a test sample (events that are not used in the training phase) that has sufficiently large statistics. 
Therefore, a \textit{k-Fold Cross-Validation} method is exploited, where k=5, so that the \SI{80}{\%} of available MC statistics is used for the training while 
\SI{20}{\%} for the testing, as described below.
All samples, including MC systematics samples and data (currently only in CRs), are divided into five approximately equal size groups using pseudo-random numbers. %\footnote{A pseudo-random number is generated for each event using the \texttt{C++ srand} generator with the sum of the dataset \texttt{RunNumber}, \texttt{DSID} and \texttt{EventNumber} as seed.}
%, which ensures that the same event in nominal samples %and systematics samples%\footnote{For example, the sample with varied jet energy resolution.} 
%is assigned in the same group.  
All events in each group have assigned the same integer pseudo-random number from 1 to 5 so that
five equivalent GBDT models are trained using four groups of nominal MC samples to test the stability of the training. 
Each training uses different combination of four groups out of five. 
The remaining one group is used as a test sample.
Each of five GBDTs is evaluated on events with the assigned pseudo-random number
that is not assigned to the training events of that GBDT. \\
\Cref{tab:BDTparam} shows the values for configuration options of the BDT method.
They are chosen to counteract overtraining and have an optimal performance.
\begin{table}[!htbp]
	\small
	\centering 
	\begin{tabular}{cc}
		\toprule
		Option & Value for \Dthree \\
		\hline
		NTrees & 800 \\ 
		MinNodeSize & 2\% \\
		BoostType & Grad \\
		Shrinkage & 0.05 \\
		UseBaggedBoost  & True \\
		BaggedSampleFraction & 0.6\\
		nCuts  & 200 \\
		MaxDepth &  2\\
		NegWeightTreatment & IgnoreNegWeightsInTraining\\
		\bottomrule
	\end{tabular}
	\caption{
	Used values for configuration options of the TMVA method BDT~\cite{TMVA}. 
}%
\label{tab:BDTparam}
\end{table}

\subsection {GBDT performance and overtraining checks }
An important step to validate the GBDT training is the \textit{overtraining} check, needed to 
verify if there is disagreement between the output from the training sample and the test sample and therefore to 
verify its stability. In fact, overtraining  leads  to  a  seeming  increase  in  the  classification 
performance  over  the objectively  achievable  one,  if  measured  on  the  training  sample, 
 and  to  an  effective performance decrease when measured with an independent test sample.
A convenient way to detect overtraining and to measure its impact is therefore to compare the
performance results between training and test samples. \Cref{fig:separation:SR3:ROC}
present the Receiver Operating  Characteristic (ROC) curves for each GBDT output score in
the signal region, while \Cref{fig:separation:SR3:GBDTsig,fig:separation:SR3:GBDTbkg} show
the GBDT output score distributions for signal and background samples, comparing results
between training and test samples.
No significant overtraining is detected. The five GBDT output scores used to
built discriminant variables are compared in~\Cref{fig:separation:GBDT}. 
The results of the five GBDTs are in agreement within the statistical uncertainties indicating the 
good stability of the trained GBDTs.
 %The GBDTs suffer from low background MC statistics, however results seem to be fairly stable within the statistical uncertainties. 
\\Input variables importance for each GBDT are presented in~\Cref{tab:D3importance}. The importance is evaluated as the total separation gain that this variable had in the decision trees (weighted by the number of events). It is normalized to all variables together, which have an importance of 1.
For most of the variables, the spread of importance values across the five GBDTs is below 3\%, indicating again a good stability of the trained GBDTs.
%\enlargethispage{5cm}
\begin{table}[!htbp]
	\small
	\centering
	\begin{tabular}{cccccc}
		\toprule
		Variable & GBDT \#1 & GBDT \#2 & GBDT \#3 & GBDT \#4 & GBDT \#5 \\
		\midrule
		$m_{q\ell\ell}$  &  0.1807  &  0.1758  &  0.1795  &  0.1799  &  0.1767  \\ 
		$\Delta R(\mu^{soft},Z)$  &  0.1613  &  0.1597  &  0.1603  &  0.1555  &  0.1616  \\ 
		$\Delta R(t_{\text{SM}},t_{\text{FCNC}})$  &  0.161  &  0.1635  &  0.1598  &  0.1634  &  0.1635  \\ 
		$m_{b\ell\nu}$  &  0.1536  &  0.1566  &  0.1565  &  0.1642  &  0.1478  \\ 
		$\Delta R(\ell,Z)$  &  0.1359  &  0.1433  &  0.1399  &  0.1378  &  0.1446  \\ 
		$\frac{\mu^{soft} ID p_{T}}{SMT\,jet\,Sum p_{T} Trk}$  &  0.1216  &  0.1179  &  0.122  &  0.1203  &  0.1216  \\ 
		$N\,b\,jets$  &  0.08582  &  0.08308  &  0.08205  &  0.07875  &  0.08431  \\ 
		\bottomrule
	\end{tabular}
	\caption{
	Input variables importance in each GBDT used to build the \Dthree discriminant.
}%
\label{tab:D3importance}
\end{table}

\begin{figure}[htbp]
	\centering
	\begin{tabular}{ccc}
		\includegraphics[width=.3\textwidth]{Chapters/CH6/figures/SR3_UsingSMT/BDT/ROC_Fold1} &
		\includegraphics[width=.3\textwidth]{Chapters/CH6/figures/SR3_UsingSMT/BDT/ROC_Fold2} &
		\includegraphics[width=.3\textwidth]{Chapters/CH6/figures/SR3_UsingSMT/BDT/ROC_Fold3} \\ 
		\multicolumn{3}{c}{
			\includegraphics[width=.3\textwidth]{Chapters/CH6/figures/SR3_UsingSMT/BDT/ROC_Fold4}
			\includegraphics[width=.3\textwidth]{Chapters/CH6/figures/SR3_UsingSMT/BDT/ROC_Fold5}} \\
	\end{tabular}
	\caption{ The ROC curves for each of five GBDTs trained in SR3\tZc to build the \Dthree discriminant. 
		Comparing results between training and test samples.
	}%
	\label{fig:separation:SR3:ROC}
\end{figure}

\begin{figure}[htbp]
	\centering
	\begin{tabular}{ccc}
		\includegraphics[width=.29\textwidth]{Chapters/CH6/figures/SR3_UsingSMT/BDT/GBDT_signal_Fold1} &
		\includegraphics[width=.29\textwidth]{Chapters/CH6/figures/SR3_UsingSMT/BDT/GBDT_signal_Fold2} &
		\includegraphics[width=.29\textwidth]{Chapters/CH6/figures/SR3_UsingSMT/BDT/GBDT_signal_Fold3} \\ 
		\multicolumn{3}{c}{
			\includegraphics[width=.29\textwidth]{Chapters/CH6/figures/SR3_UsingSMT/BDT/GBDT_signal_Fold4}
			\includegraphics[width=.29\textwidth]{Chapters/CH6/figures/SR3_UsingSMT/BDT/GBDT_signal_Fold5}} \\
	\end{tabular}
	\caption{ The FCNC \tZc \ttbar decay signal GBDT output score distributions for each of the five GBDTs trained in SR3\tZc to build the \Dthree discriminant. 
		Comparing results between training and test samples.
	}%
	\label{fig:separation:SR3:GBDTsig}
\end{figure}

\begin{figure}[!htbp]
	\centering
	\begin{tabular}{ccc}
		\includegraphics[width=.29\textwidth]{Chapters/CH6/figures/SR3_UsingSMT/BDT/GBDT_background_Fold1} &
		\includegraphics[width=.29\textwidth]{Chapters/CH6/figures/SR3_UsingSMT/BDT/GBDT_background_Fold2} &
		\includegraphics[width=.29\textwidth]{Chapters/CH6/figures/SR3_UsingSMT/BDT/GBDT_background_Fold3} \\ 
		\multicolumn{3}{c}{
		\includegraphics[width=.29\textwidth]{Chapters/CH6/figures/SR3_UsingSMT/BDT/GBDT_background_Fold4}
		\includegraphics[width=.29\textwidth]{Chapters/CH6/figures/SR3_UsingSMT/BDT/GBDT_background_Fold5}} \\
	\end{tabular}
	\caption{ The background GBDT output score distributions for each of the five GBDTs trained in SR3\tZc to build the \Dthree discriminant. 
		Comparing results between training and test samples.
	}%
	\label{fig:separation:SR3:GBDTbkg}
\end{figure}

\begin{figure}[!htbp]
	\centering
		\subfigure[]{\includegraphics[width=.45\textwidth]{Chapters/CH6/figures/SR3_UsingSMT/BDT/GBDT_signal}\label{subfig:separation:GBDTsig}}\qquad
		\subfigure[]{\includegraphics[width=.45\textwidth]{Chapters/CH6/figures/SR3_UsingSMT/BDT/GBDT_background}\label{subfig:separation:GBDTbkg}}
	\caption{ The GBDT output score distributions for \subref{subfig:separation:GBDTsig} signal events and \subref{subfig:separation:GBDTbkg} background events, in the test samples.
		The five trained GBDTs are compared in the signal region. 
	}%
	\label{fig:separation:GBDT}
\end{figure}

\clearpage
\section {The alternative selection using the \Pqc-tagger $\mathrm{DL1r_{c}}$}
\label{sec:other_selection}
The \Pqc-tagger $\mathrm{DL1r_{c}}$ (see \Cref{sec:object:cjet}) has been investigated for SR3\tZc as an alternative to SMT, already discussed in \Cref{sec:sel:sr3}.
The requirements for this selection are the following:
\begin{itemize}
	\item At least two jets satisfying the requirements described in \Cref{sec:object:jet}. 
	\item Exactly one \Pqb-jet satisfying the requirements in \Cref{sec:object:bjet}. 
	\item At least one \Pqc-jets satisfying the requirements described in \Cref{sec:object:cjet}.
	%\item The \Pqc-jet is associated to \mtopfcnc in the $\chi^2$ calculation in \Cref{eq:chi2tt}.  
	\item No requirements on the masses of both the FCNC and the SM top-quark candidates are applied. 
\end{itemize}
This selection is summarized in Table~\ref{tab:sel:sr3_dl1rc} together with the selection using SMT for comparison. Kinematic distributions are presented in \Cref{app:SRs:SR3}.     
\begin{table}[!h]
	\centering
	\small
	\begin{tabular}{ P{5.75cm} | P{5cm} }
		\toprule
		\multicolumn{2}{c}{Common selections} \\
		\midrule
		\multicolumn{2}{c}{Exactly 3 leptons with $|\eta| < 2.5$ and $\pT(\ell_1)> \SI{27}{\GeV}$, $\pT(\ell_2)> \SI{15}{\GeV}$, $\pT(\ell_3)> \SI{15}{\GeV}$} \\
		\multicolumn{2}{c}{$\ge$ 1 \OSSF pair, with $|\mll - \SI{91.2}{\GeV}| < \SI{15}{\GeV}$} \\
		\multicolumn{2}{c}{$\ge$ 2 jets with $|\eta| < $ 2.5} \\
		\midrule
		\midrule
		SR3\tZc using SMT 			  &  SR3\tZc using \DLrc \\
		\midrule
		$\le$ 2 \Pqb-tagged jets 	& =1  \Pqb-jet             \\
		OS($\mu^{soft}$,$\ell_W$) & --								\\
		$\ge$ 1 SMT jet  				 & --							   \\
		--  									 & $\ge$ 1 \Pqc-tagged jet \\
		\bottomrule
	\end{tabular}
	\caption{Overview of the requirements applied to select events in the Signal Region with \DLrc}
	\label{tab:sel:sr3_dl1rc}
\end{table}    
\\The event yields for this selection are shown in Table~\ref{tab:yields:sr3_dl1rc}.
\begin{table}[!h]
	\centering
	\small
		\begin{tabular}{|l|l|}
	\hline
	\textbf{Sample}                  			 & \textbf{Total yield}     \\
	\hline
	ZZ+LF                 & 0.71 $\pm$ 0.07          \\   
	ZZ+HF                 & 5.31 $\pm$ 0.13         \\   
	WZ+LF                  & 2.18 $\pm$ 0.12         \\   
	WZ+HF                  & 25.02 $\pm$ 0.39         \\   
	VV (2l)                 					  & 0.05 $\pm$ 0.04                             \\   
	WtZ                       					  & 12.39 $\pm$ 0.49                                \\   
	$t\bar{t}W$             					  & 2.04 $\pm$ 0.12                             \\   
	$t\bar{t}Z$ (2l)       					      & 0.02 $\pm$ 0.02                             \\   
	$t\bar{t}Z$           &69.49 $\pm$ 0.61   \\   
	Wt                      					  & 0.00 $\pm$ 0.00                           \\   
	tZ                      					  & 13.82 $\pm$ 0.28                            \\     
	$t\bar{t}$             						  & 3.66 $\pm$ 0.37                            \\   
	Z+jets                 						  & 1.32 $\pm$ 0.58                            \\   
	4 tops                 						  & 0.09 $\pm$ 0.01                           \\   
	3 tops                 						  & 0.02 $\pm$ 0.00                           \\   
	VVV                     					  & 0.22 $\pm$ 0.02                           \\   
	VH                      					  & 0.00 $\pm$ 0.00                           \\   
	$t\bar{t}H$             					  & 2.63 $\pm$ 0.05                           \\   
	$t\bar{t}WW$          					      & 0.16 $\pm$ 0.04                           \\   
	\hline                                                                    
	Total bkg.              					  &  139.13 $\pm$ 1.17                          \\       
	\hline                                                                     
	FCNC $t\bar{t}$(cZ)   					      &  21.94 $\pm$ 0.39                \\
	FCNC (c)tZ              				      &  1.21 $\pm$ 0.03                          \\
	%		\hline                                                                    
	%		$S/\sqrt{B}$ (ctZ)      					  &  1.96 $\pm$ 0.02                      \\
	\hline
\end{tabular}
	\caption{Total event yields for the SR3\tZc selection using the \Pqc-tagger \DLrc.}
\label{tab:yields:sr3_dl1rc}
\end{table}  
\newline \noindent Comparing the two event yields in Table~\ref{tab:yields:sr3} with Table~\ref{tab:yields:sr3_dl1rc} it is possible to see that using \DLrc the number of signal events is significantly larger than using SMT since the semi-leptonic decay of heavy hadrons is limited by the branching ratio (20\%). 
To increase the signal acceptance in the SMT selection, not only events containing exactly one \Pqb-jet, but also events containing zero or two \Pqb-jets have been considered.
However, taking into account the SMT selection with only one b-jet, it is also possible to see that SMT has a better discrimination of backgrounds manly due to a better light-jet rejection.\\
To choose the best c-tagger for this analysis, one can compare the values of \ssplusb for each bin of the GBDT discriminant, as it can be seen in Table~\ref{tab:yields:sr3_smt_bdt} for the SMT selection, in Table~\ref{tab:yields:sr3_dl1rc_bdt} for \DLrc selection, and Figure~\ref{fig:yields:sr3_bdt}. A new GBDT training has been performed for the \DLrc selection. The GBDT output for SMT was already presented in \Cref{sec:separation}, while for \DLrc it will be presented in \Cref{sec:separation_all} together with the discriminants for all the other Signal Regions defined.
For the \DLrc selection, in the last three bins of the GBDT discriminant, there are
12.3 events of signal and 9.3 events of background 
which corresponds to more signal events and 10\% background events of the whole SR3\tZc with SMT.\\
The GBDT output for SMT was already presented in \Cref{sec:separation}, while for \DLrc it will be presented in \Cref{sec:separation_all} together with the discriminants for all the other Signal Regions defined.

\FloatBarrier
\begin{figure}[!htbp]
	\centering
	\subfigure[SMT selection]{
		\includegraphics[width=.45\textwidth]{Chapters/CH6/figures/GBDT_smt}
		\label{fig:yields:sr3_smt_bdt}} \qquad
	\subfigure[\DLrc selection]{
		\includegraphics[width=.45\textwidth]{Chapters/CH6/figures/GBDT_dl1rc}
		\label{fig:yields:sr3_dl1rc_bdt}} \\
	\caption{Values of \ssplusb for each bin of the GBDT 
		discriminant~\subref{fig:yields:sr3_smt_bdt} for the SMT selection 
		and~\subref{fig:yields:sr3_dl1rc_bdt} for the \DLrc selection. 	}%
	\label{fig:yields:sr3_bdt}
\end{figure}
\FloatBarrier
\begin{table}[!h]
	\centering
	\small
\begin{tabular}{|l|c|c|c|c|c|} 
	\hline 
	Sample 			       & Bin 6            & Bin 7            & Bin 8             & Bin 9           & Bin 10          \\ 
	\hline                 
	Others & 0.46 $\pm$ 0.42 & 0.91 $\pm$ 0.63 & 0.04 $\pm$ 0.05 & 0.02 $\pm$ 0.05 & 0.00 $\pm$ 0.05 \\ 
	$Z$ + jets & 0.18 $\pm$ 0.17 & 0.00 $\pm$ 0.17 & 0.26 $\pm$ 0.23 & 0.40 $\pm$ 0.28 & 0.21 $\pm$ 0.36 \\ 
	$t\bar{t}+tW$ & 1.07 $\pm$ 0.28 & 0.61 $\pm$ 0.24 & 0.43 $\pm$ 0.23 & 0.59 $\pm$ 0.22 & 0.16 $\pm$ 0.21 \\ 
	$tZq$ & 2.80 $\pm$ 0.13 & 2.08 $\pm$ 0.12 & 2.10 $\pm$ 0.11 & 1.48 $\pm$ 0.10 & 1.03 $\pm$ 0.08 \\ 
	$VV$ + HF & 6.11 $\pm$ 0.19 & 4.65 $\pm$ 0.20 & 2.85 $\pm$ 0.14 & 1.70 $\pm$ 0.11 & 0.82 $\pm$ 0.09 \\ 
	$VV$ + LF & 2.67 $\pm$ 0.13 & 1.56 $\pm$ 0.10 & 0.62 $\pm$ 0.07 & 0.15 $\pm$ 0.04 & 0.06 $\pm$ 0.01 \\ 
	$t\bar{t}H+t\bar{t}W$ & 0.55 $\pm$ 0.05 & 0.40 $\pm$ 0.04 & 0.25 $\pm$ 0.04 & 0.15 $\pm$ 0.03 & 0.06 $\pm$ 0.02 \\ 
	$t\bar{t}Z+tWZ$ & 7.83 $\pm$ 0.23 & 6.43 $\pm$ 0.21 & 4.65 $\pm$ 0.18 & 3.12 $\pm$ 0.15 & 1.62 $\pm$ 0.12 \\ 
	\hline 
	Total bkg & 21.67 $\pm$ 0.63 & 16.65 $\pm$ 0.77 & 11.21 $\pm$ 0.43 & 7.60 $\pm$ 0.42 & 3.97 $\pm$ 0.45 \\ 
	\hline 
	Signal & 1.69 $\pm$ 0.03 & 2.02 $\pm$ 0.03 & 2.31 $\pm$ 0.03 & 2.66 $\pm$ 0.03 & 2.93 $\pm$ 0.03 \\ 
	\hline 
	$S/B$ & 0.08 & 0.12 & 0.21 & 0.35 & 0.74 \\ 
	$S/\sqrt{S+B}$ & 0.35 & 0.47 & 0.63 & 0.83 & 1.12 \\ 
	\hline 
\end{tabular} 
	\caption{Values of \ssplusb for the last five bins of the GBDT discriminant for the SMT selection.}%
\label{tab:yields:sr3_smt_bdt}
\end{table}

\begin{table}[!h]
	\centering
	\small
	\begin{tabular}{|l|c|c|c|c|c|} 
		\hline 
		Sample 			       & Bin 6            & Bin 7           & Bin 8           & Bin 9           & Bin 10          \\ 
		\hline                 
		Others 			       & 0.03  $\pm$ 0.03 & 0.01 $\pm$ 0.03 & 0.02 $\pm$ 0.03 & 0.00 $\pm$ 0.03 & 0.00 $\pm$ 0.03 \\
		$Z$ + jets             & 0.36  $\pm$ 0.29 & 0.06 $\pm$ 0.07 & 0.00 $\pm$ 0.07 & 0.21 $\pm$ 0.24 & 0.00 $\pm$ 0.07 \\
		$t\bar{t}+tW$          & 0.64  $\pm$ 0.15 & 0.17 $\pm$ 0.08 & 0.16 $\pm$ 0.08 & 0.14 $\pm$ 0.07 & 0.17 $\pm$ 0.09 \\
		$tZq$ 			       & 1.80  $\pm$ 0.10 & 1.37 $\pm$ 0.08 & 0.76 $\pm$ 0.06 & 0.46 $\pm$ 0.04 & 0.27 $\pm$ 0.03 \\
		$VV$ + HF              & 2.51  $\pm$ 0.13 & 1.86 $\pm$ 0.12 & 0.97 $\pm$ 0.07 & 0.53 $\pm$ 0.06 & 0.19 $\pm$ 0.03 \\
		$VV$ + LF              & 0.25  $\pm$ 0.04 & 0.24 $\pm$ 0.05 & 0.08 $\pm$ 0.02 & 0.06 $\pm$ 0.03 & 0.00 $\pm$ 0.00 \\
		$t\bar{t}H+t\bar{t}W$  & 0.36  $\pm$ 0.04 & 0.37 $\pm$ 0.04 & 0.30 $\pm$ 0.03 & 0.17 $\pm$ 0.03 & 0.10 $\pm$ 0.02 \\
		$t\bar{t}Z+tWZ$        & 6.77  $\pm$ 0.22 & 4.53 $\pm$ 0.18 & 2.60 $\pm$ 0.14 & 1.38 $\pm$ 0.09 & 0.66 $\pm$ 0.07 \\
		\hline                                                                                                           
		Total bkg              & 12.74 $\pm$ 0.43 & 8.60 $\pm$ 0.26 & 4.88 $\pm$ 0.20 & 2.97 $\pm$ 0.28 & 1.40 $\pm$ 0.14 \\
		\hline                                                                                                           
		Signal                 & 2.23  $\pm$ 0.12 & 2.98 $\pm$ 0.14 & 3.69 $\pm$ 0.16 & 4.07 $\pm$ 0.16 & 4.54 $\pm$ 0.17 \\
		\hline                 
		$S/B$                  & 0.17             & 0.35            & 0.76            & 1.37            & 3.24            \\
		$S/\sqrt{S+B}$         & 0.58             & 0.88            & 1.26            & 1.53            & 1.86       \\
		\hline
	\end{tabular} 
	\caption{Values of \ssplusb for the last five bins of the GBDT discriminant for the \DLrc selection.}%
	\label{tab:yields:sr3_dl1rc_bdt}
\end{table}


