
% this file is called up by thesis.tex
% content in this file will be fed into the main document

% ----------------------- introduction file header -----------------------
\chapter{Extraction of the limit on the branching ratio $\mathrm{\Pqt\rightarrow\PZ\Pqc}$ }
\label{chapter:Statistical_analysis}

% ----------------------- paths to graphics ------------------------

% the code below specifies where the figures are stored
\graphicspath{Chapters/CH8/figures/}

% ----------------------------------------------------------------------
% ----------------------- introduction content -------------------------
% ----------------------------------------------------------------------
In this section the statistical analysis for the extraction of the limit on the branching ratio $t \rightarrow Zc$ is presented.
In \Cref{sec:systematic} a discussion on the systematic uncertainties included in this study is reported.
In \Cref{sec:stat:strategy} the fit strategy for the \tZc coupling is presented.
In \Cref{sec:stat:summary} a summary of how the fit is performed is reported.  
In \Cref{sec:stat:tzc} the results of the fit for the \tZc coupling extraction is described. 

%%-------------------------  SYSTEMATICS %%
\FloatBarrier
%
\section{Systematic uncertainties}
\label{sec:systematic}
Many sources of systematic uncertainties are considered in the search
for FCNC $tZc$ interaction and all of them will be reported in this section. 
Systematic uncertainties that cause variations on the signal acceptance, the background rates, and the shape of the distributions that are fed to the fit are considered. They are evaluated following the common prescriptions and the standard ATLAS procedures. Systematics uncertainties from various sources are considered for the normalisation and shape of the individual backgrounds.

\subsection {Sources of systematic uncertainties}
\label{sec:systematic:sources}
Systematic uncertainties due to residual differences between data and
Monte Carlo simulations, for jet, electron and muon reconstruction
after calibration, as well as uncertainties on the calibration scale
factors are propagated to the event yields and observables.

\begin{itemize}
	\item \textbf{Lepton reconstruction} \\ 
	The mis-modelling of muon (electron) trigger, reconstruction,
	identification and isolation efficiencies in simulation is corrected by introducing
	scale factors derived from measured efficiencies in data. The decays
	of $Z \to \mu^+ \mu^-$ ($Z \to e^+ e^-$) are used to obtain scale
	factors as functions of the lepton kinematics. The uncertainties are
	evaluated by varying the lepton and signal selections and from the
	uncertainties in the backgrounds evaluations. \\
	For the SMT muons it was proved that the scale factors obtained for
	isolated muons are valid also for muons inside jets~\cite{SMT-INT-13TeV}. 
	\item \textbf{Lepton momentum scale and resolution} \\ 
	The $Z \to ll$ process is used to measure the lepton momentum scale
	and resolution. Calibration factors and associated uncertainties are
	derived to match the simulation to observed distributions in collision
	data. The effect of momentum scale uncertainties is evaluated by
	repeating the event selection after varying the lepton momentum up by
	$1\sigma$ and down by $1\sigma$. For the momentum resolution
	uncertainties, the event selection is repeated by smearing the lepton
	momentum~\cite{PERF-2013-05}.
	\item \textbf{Jet energy scale} \\ 
	The jet energy scale (JES) was derived using information from test-beam data, LHC
	collision data and simulation. The JES calibration consists of several
	steps that account for detector problems, jet reconstruction
	algorithms, jet fragmentation models, dense data-taking environment
	from high pile-up conditions and response difference between data and
	MC simulation. 
	The fractional uncertainty decreases with the $\Pt$ of the
	reconstructed jet and is rather stable in $\eta$. The JES uncertainty
	has various components according to the factors it accounts for and
	the different steps used to compute it. The jet calibration procedure
	is described in Ref.~\cite{PERF-2016-04}.
	The sources of the JES uncertainties with different effective number
	of parameters are: BJES response, detector, mixed, modelling,
	statistical, eta intercalibration, flavour composition, pile-up and
	relative non-closure. 
	\item \textbf{Jet energy resolution} \\ 
	The impact of the uncertainty on the jet energy resolution (JER) is
	evaluated by smearing the jet energy in the MC samples. 
	\item \textbf{Jet vertex tagger} \\ 
	The uncertainty for the JVT requirement is also applied. 	
	\item \textbf{Missing transverse momentum} \\ 
	Uncertainties of the soft-track component are derived from the level
	of agreement between data and MC simulation of the $\Pt$ balance
	between the hard and soft $\Etmiss$ components. Three different
	uncertainties are considered: an offset along the $\Pt$ (hard) axis,
	as well as the smearing resolution along and perpendicular to the
	$\Pt$ (hard) axis.
	\item \textbf{$b$-tagging efficiency} \\ 
	The $b$-tagging efficiencies and mis-tag rate for the taggers have
	been measured in data using the same methods as described
	in~\cite{ATLAS-CONF-2014-046,ATLAS-CONF-2014-004}. The number of NP
	used for the $b$-tagging data/MC scale factors is evaluated
	separately for $b$, $c$ and light-flavour quark jets in the MC
	samples.
	\item \textbf{$c$-tagging efficiency} \\
	Since an official calibration is not available yet, the $c$-tagging
	efficiencies are not considered in this analysis. Nevertheless,
	assuming a flat 20\% variation on the related scale factor, 
	the final expected limit will be degraded by almost 5\%.	
\end{itemize}

\newpage
% ===========================================
%\subsubsection{Monte Carlo modelling} 
% ===========================================
\noindent Systematic uncertainties on template shapes from MC modelling of various background sources are 
estimated by comparing different generators and varying parameters for the event generation.

\begin{itemize}	
	\item \textbf{Signal} \\ 
	Renormalization and factorization scale uncertainties are considered for the signal, following the
	latest prescriptions. In particular, the $\mu_r$ and $\mu_f$
	variations are included and the envelope of the variations is added 
	(called \textit{Signal $\mu_r$ and $\mu_f$}). \\
%	An additional uncertainty, comparing the left- and right-handed
%	coupling samples, is included (called \textit{Signal RH couplings}). \\
	\item \textbf{\ttbar} \\
	The effect of changing the parton shower for the generation of \ttbar events is
	included and the difference between \PythiaEight and \HERWIG7
	parton shower is added as \ttbar parton showering uncertainty (called
	\textit{\ttbar PS}). \\
	Scale, radiation and tune systematics are also included, following the
	latest prescriptions. In particular, the $\mu_r$ and $\mu_f$
	variations are included and the envelope of the variations is added 
	(called \textit{\ttbar $\mu_r$ and $\mu_f$}). The A14 tune
	variations are added (called \textit{\ttbar A14 tune (ISR)}). Finally the FSR is varied (called \textit{\ttbar FSR}). \\  
	The systematic uncertainties related to the parton distribution
	functions are taken into account for the \ttbar background (called
	\textit{\ttbar PDF}). \\
	Additionally, the uncertainty associated to the \hdamp\footnote{The \hdamp\ factor is the model parameter that controls ME/Parton Shower matching and effectively regulates the high-$\mathrm{p_{T}}$ radiation.} parameter is evaluated (NP called \textit{\ttbar hdamp})
	using the alternative sample with the \hdamp\ value increased to $3.0~\mtop$.\\
	Last but not least, an uncertainty is added to take into account the differences in the
	photon conversion and \Pqb-hadron fractions in the signal regions with
	respect to the \ttbar control region where the \ttbar background is
	controlled. 
	This uncertainty is obtained by taking the maximum
	difference in the fractions between the regions (\SI{13}{\%} for
	\Pqb-hadron and \SI{50}{\%} for photon conversions) and it's applied
	to the relevant fraction. 
	The two uncertainties are called
	\textit{\ttbar non-prompt lep. (photon conv.)} and 
	\textit{\ttbar non-prompt lep. (b-decay)}. \\
	\item \textbf{\ttZ} \\
	The effect of changing the MC generator for $t\bar{t}Z$ events was
	investigated and the difference between \aMCatNLO and \Sherpa
	prediction is included as $t\bar{t}Z$ generator systematic
	uncertainty (called \textit{\ttZ Generator}). \\
	A scale uncertainty systematic is also included, following the
	latest prescriptions. In particular, the $\mu_r$ and $\mu_f$
	variations are included and the envelope of the variations is added 
	(called \textit{\ttZ $\mu_r$} and \textit{$\mu_f$}).
	Additionally, the effects of QCD radiation for this
	sample are also taken into account through samples for different A14
	tune variations (called \textit{\ttZ QCD}).\\
	\item \textbf{\tWZ} \\
	The effect of changing the diagram removal scheme used for the generation of \tWZ events was
	investigated and the difference between the two diagram removal
	predictions is included as \tWZ generator systematic uncertainty
	(called \textit{\tWZ DR}).\\
	\vspace{\baselineskip}
	\item \textbf{\tZq} \\
	The differences of using different A14
	tune variations are accounted and considered in the fit (called
	\textit{$tZ$ QCD}).\\
	\item \textbf{Diboson} \\
	The effect of changing the MC generator for diboson events was also
	investigated and the difference between \Sherpa and \PowhegBox
	prediction is included as diboson generator uncertainty. This
	uncertainty is split into two components: light- and heavy-flavour
	(called \textit{\VVLF Generator} and \textit{\VVHF Generator}) using truth information. \\
	An uncertainty depending on the jet multiplicity is also included for
	the diboson samples with the separation by light- and heavy-flavour as
	well. Therefore, an uncertainty of \SI{25}{\%} 
	(called \textit{\VVLF N Jet} and similarly for \VVHF) is added in quadrature
	per jet in each jet multiplicity resulting into 
	5 NP (= 1, = 2, = 3, = 4, $\geq$ 5 jets). This uncertainty is also known as \textit{Berends scaling} \\ 
	\item \textbf{MC statistics} \\ 
	The uncertainty due to the limited size of the various MC samples is also included.\\
\end{itemize}

% ===========================================
%\subsubsection{Background rate uncertainty} 
% ===========================================
\noindent The uncertainties on the background rate uncertainties are summarised in
\cref{tab:syst-crosssections} and described in the following:
\begin{itemize}
	\item \textbf{\ttbar} \\
	For the $t\bar{t}$ process, an uncertainty of \SI{5.5}{\%} on the normalisation is applied. Since this analysis targets final states with three prompt leptons (see \Cref{tab:sel:srs}), the $t\bar{t}$ process is a fake background.
	Therefore a normalisation factor is added to the fit as a free-floating parameter to better control the event rate.
	\item \textbf{\ttV} \\
	For $t\bar{t}Z$ and $t\bar{t}W$ backgrounds, the uncertainty on the normalisation is taken from~\cite{ATL-COM-PHYS-2018-140}, where a \SI{12}{\%} theory uncertainty is quoted.
	\item \textbf{\tWZ} \\
	For the $tWZ$ background, an uncertainty of \SI{30}{\%} is applied to the normalisation~\cite{Bylund:2016qau}.
	\item \textbf{\tZq} \\
	For the $tZq$ process, an uncertainty of \SI{14}{\%} on the normalisation is applied following the recent results from the $tZq$ observation~\cite{Aad:2020wog}.
	\item \textbf{Diboson} \\
	For diboson background, the normalization uncertainty is taken from ATLAS results~\cite{STDM-2018-03}. The uncertainties applied are \SI{20}{\%} for the light diboson component and \SI{30}{\%} for the heavy diboson component. 
	\item \textbf{\Zjets} \\
	Concerning the \Zjets processes, an uncertainty on the normalisation is applied with a value of \SI{100}{\%} allowing the constraint of this value by the fit.
	\item \textbf{\ttH} \\
	Normalisation uncertainty of \SI{15}{\%} is applied on the \ttH background~\cite{Demartin:2016axk}. 
	\item \textbf{Other background processes} 
	A conservative overall normalization uncertainty of \SI{50}{\%} is applied
	on the remaining minor backgrounds (\ttt,\tttt,\VVV,\VH and \ttWW). 
	These background components are typically well below of \SI{1}{\%} in the SRs.
\end{itemize}

\begin{table}[t]
	\small
	\centering
	\begin{tabular}{l c }
		\toprule
		\textbf{Process} & Uncertainty \\
		\midrule
		\ttbar & \SI{5.5}{\%} \\
		\ttV    & \SI{12}{\%} \\
		\tWZ & \SI{30}{\%} \\
		\tZq & \SI{14}{\%} \\
		\VVLF & \SI{20}{\%} \\
		\VVHF & \SI{30}{\%} \\
		\Zjets & \SI{100}{\%} \\
		 \ttH    & \SI{15}{\%} \\
		Other (\ttt,\tttt,\VVV,\VH and \ttWW) & \SI{50}{\%} \\		
		\bottomrule
	\end{tabular}
	\caption{Uncertainties on the normalisation of all background processes.}
	\label{tab:syst-crosssections}
\end{table}
\FloatBarrier
% ===========================================
\noindent The uncertainty in the combined 2015-2018 integrated luminosity is
\SI{1.7}{\%}. It is derived, following a methodology similar to that
detailed in Ref.~\cite{DAPR-2013-01}, from the calibration of the
luminosity scale using $x-y$ beam-separation scans.\\
The uncertainty of the pile-up reweighting in simulated samples is included.

\subsection {Acceptance and shape uncertainties}
\label{sec:systematic:treatment}

The systematic uncertainties discussed above cause variations on the signal
acceptance, the background rates, and the shape of the distributions
that are fed to the fit.
We denote the relative variation of the acceptance of process $j$
due to a systematic source $i$ $\alpha_{ij+}$ and $\alpha_{ij-}$ for a
positive or negative variation of the systematic uncertainty.		\\
The MC statistics is related to the statistical uncertainty in each
bin of the of the distributions that are used in the fit,
so it can change the shape of the distributions. 
For all other systematics listed, rate and shape differences are taken
into account. \\
\subsubsection{Symmetrization of shape uncertainties} 
For the shape uncertainties included in the fit, the templates are symmetrised by
taking half the difference between the up and down variation around
the nominal template. For the bins where both the initial up and down
variations are on the same side from the nominal, the largest
variation is symmetrically assigned for the final templates.\\
Concerning all systematic uncertainties containing up and down
systematic variations, these are symmetrised by 
($\alpha_{ij-}$ + $\alpha_{ij+}$)/2 = $\alpha_{ij-}$ = $\alpha_{ij+}$. 
For the systematic uncertainties with one sided systematics, these
are symmetrised by defining $\alpha_{ij-}$ = $\alpha_{ij+}$.\\
\subsubsection{Smoothing and Pruning} 
Some fluctuations in individual bins can be caused
by the low statistics shape systematic uncertainties. A smoothing of
systematic shape uncertainties is performed using the default
smoothing algorithm available through the \texttt{TRExFitter} package
to average adjacent bins to remove statistical fluctuations. \\
Different levels of pruning are applied for shape
and normalisation. A value of \prunenorm and \pruneshape were used for
normalisation and shape, respectively. 


\section {Fit strategy}
\label{sec:stat:strategy}

In order to extract the \tZc coupling, a binned maximum
likelihood $L(\mu,\theta)$ fit is performed using the MC templates for both signal and
background predictions.The $L(\mu,\theta)$ is constructed as a product of Poisson
probability terms over all bins in each considered distribution, and Gaussian constraint terms
for $\theta$, a set of nuisance parameters that parametrize effects of systematic uncertainties on the signal and background expectations.\\
The software framework used for performing the fit is \texttt{TRExFitter}~\cite{TRexfitter}.
This combines the functionalities of
\texttt{RooFit}~\cite{Verkerke:2003ir} and
\texttt{RooStats}~\cite{Moneta:2010pm} and is designed to build
probability density functions that are automatically fit to data and
interpreted with statistical tests.
The likelihood function comprises histogram bins from both SRs and
CRs.

\paragraph{Regions} The regions included in the fit, as well as the
distributions that are fitted together, are summarised in \Cref{tab:fitregions}.
The definitions of the various regions are shown in
\Cref{tab:sel:srs} for the SRs and in \Cref{tab:bkg:crs} for the CRs. 
Some regions are used to control the overall normalisation of various
backgrounds.

\begin{table}[htbp]
	\small
	\centering
	\begin{tabular}{cc}
		\toprule
		\multicolumn{2}{c}{\tZc coupling extraction} \\
		\midrule
		Region & Distribution \\
		\midrule
		SR1 \tZc & \Done  \\
		SR2 \tZc & \DtwoC  \\
		SR3 \tZc & \Dthree  \\
		Side-band CR1 \tZc & \Done  \\
		Side-band CR2 & \DtwoC  \\
		\ttZ CR & Leading lepton \pt  \\
		\ttbar CR & Leading lepton \pt  \\
		\bottomrule
	\end{tabular}
	\caption{
	Overview of the regions included in the fits.}%
\label{tab:fitregions}
\end{table}

\paragraph{Inputs} The inputs to the fit consist of binned
distributions, including the signal and all background sources.
Additionally, for each MC sample, separate templates that take into
account the systematic variations discussed in \Cref{sec:systematic}
are created and included in the fit. 

\paragraph{POI} The global likelihood function describing the
agreement between data and prediction as a function of the parameter
of interest (POI) and the set of nuisance parameters describing the
effect of the corresponding systematic uncertainty sources is
constructed and fitted. The POI is the signal strength parameter, $\mu$,
a multiplicative factor for the number of signal events normalised to a reference branching ratio
$\mathrm{\BR_{\text{ref}}(\Pqt\rightarrow\PZ\Pqc)=0.024\%}$.
The relationship between $\mu$ and the corresponding $\mathrm{BR (\Pqt\rightarrow\PZ\Pqc)}$  is
\begin{equation}
\mathrm{\mu=\frac{\BR(\Pqt\rightarrow\PZ\Pqc)(1-\BR(\Pqt\rightarrow\PZ\Pqc))}{\BR_{\text{ref}}(\Pqt\rightarrow\PZ\Pqc)(1-\BR_{\text{ref}}(\Pqt\rightarrow\PZ\Pqc))}}
\end{equation}
%is defined as the ratio between the measured cross-section and the
%theoretical prediction. \LDnote{}{How is the POI defined?}

\paragraph{Systematic uncertainty NPs} The impact of the systematic
uncertainties on the extracted $\mu$ is estimated as follows. \\
A nuisance parameter (NP) is associated to each systematic uncertainty.
These NPs have a central value and an associated pre-fit uncertainty.
The fit is able to change the central value of the NPs (called \textit{pull})
and the uncertainty on the NP can change (this is called \textit{constraint}
if the uncertainty becomes smaller), to better describe the data.
%The change in each NP central value and its uncertainty are shown as a dot (central
%value) and a black line (uncertainty).
%If no pull and no constraint are present, the black dot is at 0 and
%the black line at $\pm 1$.
%This part of the figure helps in understanding what the fit learns
%about the systematic uncertainties from data.\\
To understand the impact of the NPs on the extracted $\mu$, the
following procedure is used:
several fits are performed, each fit separately for each NP,
changing the central value of each NP up and down by the pre-fit
(post-fit) $\pm 1 \sigma$ uncertainties,
fixing it to that value and performing the fit to $\mu$.
The difference between this $\mu$ and the one extracted from the
standard fit, $\Delta\mu$, is the pre-fit (post-fit) impact on $\mu$.
The values of $\Delta\mu$ are shown in the so called ranking plot that 
helps to understand the size of the effect that
the uncertainty has on the signal strength. \\
%To prevent statistical fluctuations increasing the systematic uncertainties
%and wasted computation time on fitting insignificant NPs,
%systematic uncertainties are pruned from the fit. 
%A value of \prunenorm and \pruneshape are used for normalisation and shape, respectively. 

\paragraph{Statistical uncertainty NPs} The same procedure is applied
to the so called $\gamma$ parameters, which represent the background
statistical uncertainty in each bin of the input distributions.
There is therefore one gamma parameter per bin of each input distribution. 

\paragraph{Background treatment in the fit} The \ttZ and \tWZ backgrounds are merged in one template as well as
\ttbar and $\mathrm{Wt}$. Two templates are included in the fit for diboson process that correspond to the heavy and light components (\VVHF and \VVLF).
Separate templates are used for the remaining background sources.
Since this analysis targets final states with three prompt leptons (see \Cref{tab:sel:srs}), the \ttbar and the $\mathrm{Wt}$ processes are a fake background. Therefore the normalisation of \ttbar+$\mathrm{Wt}$ background is free floating in the fit meaning that an unconstrained NP is associated to the corresponding template, namely $\mu_{\ttbar+\Wt}$.
All other backgrounds have pre-fit normalizations with uncertainties (see \Cref{sec:systematic:sources}).

\clearpage
\FloatBarrier

\section{Summary of fits}
\label{sec:stat:summary}
For the extraction of the \tZc coupling, the following fits
are presented:
\begin{description}
	\item[Background-only fit in CRs:] in this fit only CRs are used. Data are
	used. The signal templates are not included in the fit. This fit is used to estimate
	the preliminary background.
	\item[Signal+Background fit in SR+CRs with pseudo-data:] in this fit both SRs and CRs are used.
	Data are used in the CRs. In the SRs, pseudo-data are constructed
	using post-fit background normalisations from the B-only fit in CRs. In the following, this dataset is called 'realistic Asimov'. This fit is used to extract the expected upper limit.
	\item[Signal+Background fit in SR+CRs with data:] in this fit both SRs and CRs are used. Data are used in both 
	SRs and CRs. This fit is used to extract the \tZc coupling, therefore the observed and the expected upper limit.
\end{description}
These fits are summarised in the following table:

\begin{table}[htbp]
	\small
	\centering
	\begin{tabular}{cccccc}
		\toprule
		Setup & Fit & SRs & CRs & Template & Reference \\
		\midrule
		1 & B-only in CRs & -- & data & B  & Appendix \ref{app:stat:tzc:bonly:cr} \\
		\midrule
		2 & S+B in SRs+CRs & realistic Asimov & data & S+B  & \Cref{sec:stat:tzc:splusb:crsr} \\
		& with realistic Asimov & from Setup 1 & & & \\
		\midrule
		3 & S+B in SRs+CRs & data & data & S+B  & \Cref{sec:stat:tzc:splusb:unb} \\
		& with data & & & & \\
		\bottomrule
	\end{tabular}
	\caption{
	Overview of the fits for the extraction of the \tZc coupling.}%
\label{tab:summary_fits}
\end{table}

%% ----------------------------------------------------------------------------

\section{Signal + Background fit in SRs+CRs with realistic Asimov data}
\label{sec:stat:tzc:splusb:crsr}
To extract the expected sensitivity, an SRs+CRs S+B fit is performed. 
Data in used in CRs while in SRs an Asimov dataset is used,
constructed using the background normalisations found in Appendix \ref{app:stat:tzc:bonly:cr}.\\
Plots and tables shown in this section are the following:
\begin{itemize}
\item The value of the post-fit normalisation parameters of the free floating background is shown in \Cref{fig:stat:tzc:splusb:crsr:norm}.
\item The list of the systematic shapes that are dropped from the fit for each sample and for each region is shown in \cref{fig:stat:tzc:splusb:crsr:pruning}.
\item The pull distributions of the all nuisance parameters can be seen in \Cref{fig:stat:tzc:splusb:crsr:np:instr,fig:stat:tzc:splusb:crsr:np:model} and \Cref{fig:stat:tzc:splusb:crsr:gamma}. 
\item The correlation matrix of the nuisance parameters is shown in \Cref{fig:stat:tzc:splusb:crsr:corrmatrix}. 
\item The ranking of the nuisance parameters is shown in \Cref{fig:stat:tzc:splusb:crsr:ranking}. 
%Red and blue plots for the top five ranked NPs are shown in a
%dedicated appendix (\Cref{sec:app:fit:redblue:tzc:splusb:crsr}).
\item Event yields pre- and post-fit are shown in \Cref{tab:stat:tzc:splusb:crsr:yields:prefit,tab:stat:tzc:splusb:crsr:yields:postfit}. 
\item Pre-fit and post-fit distributions of the fitted distributions in the various regions are shown in \Cref{fig:stat:tzc:splusb:crsr:srplots:1,fig:stat:tzc:splusb:crsr:srplots:2,fig:stat:tzc:splusb:crsr:crplots:1,fig:stat:tzc:splusb:crsr:crplots:2}.
\end{itemize}
As expected, the behaviour of the fit with the realistic Asimov dataset
provides similar results to those of the B-only fit in the CRs
(Appendix \ref{app:stat:tzc:bonly:cr}). 
In fact, normalisation factors (\Cref{fig:stat:tzc:splusb:crsr:norm}) and NPs
pulls and constrains 
(\Cref{fig:stat:tzc:splusb:crsr:np:instr,fig:stat:tzc:splusb:crsr:np:model}) 
are very similar since in the CRs the expected signal is negligible
and in the SRs the Asimov data are used. 
The signal strength $\mu$ is expected to be zero for the same reason.
The fake normalization factor $\mu_{\ttbar+\Wt}$ is compatible with unity.\\
The most pulled NPs (e.g. \ttbar FSR and \VVHF Berends scaling for events with 1 jet) are
not among the highest ranked NPs (\Cref{fig:stat:tzc:splusb:crsr:ranking}). 
Slightly pulled NPs (e.g. \ttZ normalization, Other fakes norm. and \VVHF generator) have a small post-fit impact on the signal strength, around 2\%, as can be seen again in \Cref{fig:stat:tzc:splusb:crsr:ranking}. Moreover, none of the systematic uncertainties has a post-fit impact on the signal strength parameter greater than 3\%.\\
Concerning the correlations between NPs
(\Cref{fig:stat:tzc:splusb:crsr:corrmatrix}), some strong correlations
between diboson related NPs are present, as expected. This is also
true for the \ttbar normalisation and some \ttbar modeling NPs. 
Event yields pre- and post-fit, shown in \Cref{tab:stat:tzc:splusb:crsr:yields:prefit,tab:stat:tzc:splusb:crsr:yields:postfit} and distributions in \Cref{fig:stat:tzc:splusb:crsr:srplots:1,fig:stat:tzc:splusb:crsr:srplots:2,fig:stat:tzc:splusb:crsr:crplots:1,fig:stat:tzc:splusb:crsr:crplots:2} show a good agreement between the observed data and MC predictions in Control and Signal Regions.

\begin{figure}[htbp]
	\centering
	\includegraphics[width=.5\textwidth]{Chapters/CH8/figures/SPLUSB_CRSR_UsingDL1rcFullSys/NormFactors}
	\caption{Normalisation factors for the S+B \tZc fit in SRs+CRs with realistic Asimov.}%
	\label{fig:stat:tzc:splusb:crsr:norm}
\end{figure}

\begin{figure}[htbp]
	\centering
	\includegraphics[width=.85\textwidth]{Chapters/CH8/figures/SPLUSB_CRSR_UsingDL1rcFullSys/Pruning}
	\caption{Pruning of the nuisance parameters for the S+B \tZc fit in SRs+CRs with realistic Asimov.}%
	\label{fig:stat:tzc:splusb:crsr:pruning}
\end{figure}

\begin{figure}[htbp]
	\centering
	\includegraphics[width=.8\textwidth]{Chapters/CH8/figures/SPLUSB_CRSR_UsingDL1rcFullSys/NuisPar_Instrumental}
	\caption{Pulls and constraints of the instrumental nuisance parameters for the S+B \tZc fit in SRs+CRs with realistic Asimov.}%
	\label{fig:stat:tzc:splusb:crsr:np:instr}
\end{figure}

\begin{figure}[htbp]
	\centering
	\includegraphics[width=.85\textwidth]{Chapters/CH8/figures/SPLUSB_CRSR_UsingDL1rcFullSys/NuisPar_Theoretical_&_Modelling}
	\caption{Pulls and constraints of the theoretical and modeling nuisance parameters for the S+B \tZc fit in SRs+CRs with realistic Asimov.}%
	\label{fig:stat:tzc:splusb:crsr:np:model}
\end{figure}

\begin{figure}[htbp]
	\centering
	\includegraphics[width=.85\textwidth]{Chapters/CH8/figures/SPLUSB_CRSR_UsingDL1rcFullSys/Gammas}
	\caption{Gamma parameters for the S+B \tZc fit in SRs+CRs with realistic Asimov.}%
	\label{fig:stat:tzc:splusb:crsr:gamma}
\end{figure}

\begin{figure}[htbp]
	\centering
	\includegraphics[width=.95\textwidth]{Chapters/CH8/figures/SPLUSB_CRSR_UsingDL1rcFullSys/CorrMatrix}
	\caption{Correlation matrix of the nuisance paramenters for the S+B \tZc fit in SRs+CRs with realistic Asimov.}%
	\label{fig:stat:tzc:splusb:crsr:corrmatrix}
\end{figure}

\begin{figure}[htbp]
	\centering
	\includegraphics[width=.9\textwidth]{Chapters/CH8/figures/SPLUSB_CRSR_UsingDL1rcFullSys/Ranking}
	\caption{Ranking of the nuisance parameters for the S+B \tZc fit in SRs+CRs with realistic Asimov.}%
	\label{fig:stat:tzc:splusb:crsr:ranking}
\end{figure}

\FloatBarrier
\clearpage
%\global\pdfpageattr\expandafter{\the\pdfpageattr/Rotate 90}
\begin{table}[]
	\centering
	\tiny
	% NB: add to main document: 
% \usepackage{siunitx} 
% \sisetup{separate-uncertainty,table-format=6.3(6)}  % hint: modify table-format to best fit your tables
\begin{tabular}{|p{0.10\textwidth}|>{\centering}p{0.08\textwidth}|>{\centering}p{0.08\textwidth}|>{\centering}p{0.08\textwidth}|>{\centering}p{0.09\textwidth}|>{\centering}p{0.09\textwidth}|>{\centering}p{0.09\textwidth}|>{\centering\arraybackslash}p{0.09\textwidth}|}
\toprule  
& {SR1} & {SR2} & {SR3} & {Side-band CR1} & {Side-band CR2} & {\ttZ CR} & {\ttbar CR}\\
\midrule 
    \ttZ+\tWZ   & 168 $\pm$ 22 & 33 $\pm$ 7 & 82 $\pm$ 11 & 88 $\pm$ 12 & 9.1 $\pm$ 2.1 & 164 $\pm$ 22 & 14.8 $\pm$ 1.9 \\ 
\ttW   & 5.8 $\pm$ 1.0 & 3.3 $\pm$ 0.6 & 2.04 $\pm$ 0.35 & 4.3 $\pm$ 0.7 & 2.5 $\pm$ 0.5 & 2.3 $\pm$ 0.5 & 27 $\pm$ 4 \\ 
\ttH   & 6.1 $\pm$ 1.0 & 0.88 $\pm$ 0.18 & 2.6 $\pm$ 0.4 & 2.3 $\pm$ 0.4 & 0.36 $\pm$ 0.07 & 5.4 $\pm$ 0.9 & 13.8 $\pm$ 2.1 \\ 
\VVLF   & 28 $\pm$ 17 & 35 $\pm$ 13 & 2.9 $\pm$ 2.0 & 25 $\pm$ 15 & 18 $\pm$ 7 & 0.20 $\pm$ 0.22 & 0.40 $\pm$ 0.21 \\ 
\VVHF   & 140 $\pm$ 100 & 160 $\pm$ 70 & 30 $\pm$ 22 & 130 $\pm$ 80 & 69 $\pm$ 28 & 13 $\pm$ 11 & 2.3 $\pm$ 1.4 \\ 
\tZq   & 47 $\pm$ 7 & 110 $\pm$ 18 & 13.8 $\pm$ 2.3 & 20 $\pm$ 4 & 9.9 $\pm$ 1.7 & 14.6 $\pm$ 2.9 & 0.90 $\pm$ 0.15 \\ 
\ttbar+Wt   & 21 $\pm$ 4 & 32 $\pm$ 11 & 3.7 $\pm$ 1.0 & 10 $\pm$ 4 & 9.1 $\pm$ 2.7 & 3.0 $\pm$ 1.2 & 102 $\pm$ 24 \\ 
Other fakes   & 10 $\pm$ 11 & 12 $\pm$ 12 & 1.4 $\pm$ 1.6 & 3 $\pm$ 5 & 10 $\pm$ 11 & 0.00 $\pm$ 0.06 & 0.12 $\pm$ 0.14 \\ 
Other   & 2.5 $\pm$ 1.5 & 3.8 $\pm$ 2.8 & 0.48 $\pm$ 0.25 & 2.2 $\pm$ 1.6 & 0.8 $\pm$ 2.6 & 1.1 $\pm$ 0.5 & 2.9 $\pm$ 1.5 \\ 
\midrule 
Total background  & 430 $\pm$ 110 & 390 $\pm$ 80 & 139 $\pm$ 25 & 280 $\pm$ 80 & 130 $\pm$ 32 & 203 $\pm$ 27 & 164 $\pm$ 25 \\ 
\midrule 
Data   & 433 & 443 & 143 & 331 & 169 & 197 & 156 \\ 
\midrule 
Data / Bkg.   & 1.00 $\pm$ 0.25 & 1.15 $\pm$ 0.24 & 1.03 $\pm$ 0.20 & 1.18 $\pm$ 0.35 & 1.30 $\pm$ 0.34 & 0.97 $\pm$ 0.14 & 0.95 $\pm$ 0.16 \\ 
\bottomrule 
\end{tabular} 

	\caption{Pre-fit event yields in the S+B \tZc fit in SRs+CRs with realistic Asimov. \TabErrStatSys} 
	\label{tab:stat:tzc:splusb:crsr:yields:prefit}
\end{table} 

\begin{table}[]
	\centering
	\tiny
	% NB: add to main document: 
% \usepackage{siunitx} 
% \sisetup{separate-uncertainty,table-format=6.3(6)}  % hint: modify table-format to best fit your tables
\begin{tabular}{|l|S|S|S|S|S|S|S|}
\toprule  
 & {SR1tZc} & {SR2tZc} & {SR3tZc} & {Side-band CR1tZc} & {Side-band CR2} & {\ttZ CR} & {\ttbar CR}\\
\midrule 
  \ttZ+\tWZ   & 163 $\pm$  14 & 34 $\pm$  6 & 79 $\pm$  9 & 86 $\pm$  9 & 9.3 $\pm$  1.9 & 157 $\pm$  13 & 14.4 $\pm$  1.4 \\ 
  \ttW   & 5.7 $\pm$  0.9 & 3.4 $\pm$  0.6 & 2.0 $\pm$  0.4 & 4.2 $\pm$  0.7 & 2.5 $\pm$  0.5 & 2.2 $\pm$  0.4 & 26 $\pm$  4 \\ 
  \ttH   & 6.1 $\pm$  0.9 & 0.90 $\pm$  0.17 & 2.6 $\pm$  0.5 & 2.3 $\pm$  0.4 & 0.37 $\pm$  0.07 & 5.3 $\pm$  0.8 & 13.8 $\pm$  2.1 \\ 
  \VVLF   & 32 $\pm$  18 & 39 $\pm$  14 & 3.3 $\pm$  2.2 & 29 $\pm$  16 & 21 $\pm$  8 & 0.24 $\pm$  0.23 & 0.40 $\pm$  0.18 \\ 
  \VVHF   & 198 $\pm$  32 & 212 $\pm$  29 & 43 $\pm$  8 & 172 $\pm$  25 & 94 $\pm$  16 & 18 $\pm$  6 & 3.3 $\pm$  0.5 \\ 
  \tZq   & 46 $\pm$  7 & 112 $\pm$  16 & 13.9 $\pm$  2.4 & 19.6 $\pm$  3.3 & 10.1 $\pm$  1.6 & 14.4 $\pm$  2.5 & 0.91 $\pm$  0.12 \\ 
  \ttbar+Wt   & 18 $\pm$  4 & 30 $\pm$  7 & 3.2 $\pm$  0.8 & 9.5 $\pm$  2.8 & 8.6 $\pm$  1.7 & 2.5 $\pm$  0.8 & 95 $\pm$  13 \\ 
  Other fakes   & 15 $\pm$  11 & 17 $\pm$  13 & 2.1 $\pm$  1.7 & 5 $\pm$  5 & 18 $\pm$  15 & 0.005 $\pm$  0.009 & 0.18 $\pm$  0.13 \\ 
  Other   & 2.2 $\pm$  1.2 & 3.7 $\pm$  2.5 & 0.44 $\pm$  0.24 & 1.8 v 1.2 & 0.2 $\pm$  0.8 & 1.0 $\pm$  0.5 & 2.7 $pm$  1.4 \\ 
  FCNC (c)tZ   & 0.0 $pm$ 0.6 & 0.0 $pm$ 2.1 & 0.00 $pm$ 0.21 & 0.00 $pm$ 0.18 & 0.00 $pm$ 0.15 & 0.00 $pm$ 0.04 & 0.000 $pm$ 0.015 \\ 
  FCNC \ttbar(cZ)   & 0 $pm$ 10 & 0.1 $pm$ 3.2 & 0 $pm$ 4 & 0.0 $pm$ 0.7 & 0.01 $pm$ 0.34 & 0.0 $pm$ 0.6 & 0.00 $pm$ 0.06 \\ 
  \midrule 
  Total background  & 487 $pm$ 21 & 452 $pm$ 20 & 150 $pm$ 12 & 328 $pm$ 17 & 165 $pm$ 13 & 201 $pm$ 12 & 157 $pm$ 12 \\ 
  \midrule 
  Data   & 488 & 452 & 150 & 331 & 169 & 197 & 156 \\ 
  \midrule 
  Data / Bkg.   & 1.00 $pm$ 0.04 & 1.00 $pm$ 0.04 & 1.00 $pm$ 0.08 & 1.01 $pm$ 0.05 & 1.02 $pm$ 0.08 & 0.98 $pm$ 0.06 & 0.99 $pm$ 0.08 \\ 
  \bottomrule 
\end{tabular} 

	\caption{Post-fit event yields in the S+B \tZc fit in SRs+CRs with realistic Asimov. \TabErrStatSys} 
	\label{tab:stat:tzc:splusb:crsr:yields:postfit}
\end{table} 
\clearpage
\FloatBarrier
%\global\pdfpageattr\expandafter{\the\pdfpageattr/Rotate 0}

\clearpage
\begin{figure}[htbp]
	\centering
	\begin{tabular}{cc}
		\includegraphics[width=.45\textwidth]{Chapters/CH8/figures/SPLUSB_CRSR_UsingDL1rcFullSys/Plots/SR1} &
		\includegraphics[width=.45\textwidth]{Chapters/CH8/figures/SPLUSB_CRSR_UsingDL1rcFullSys/Plots/SR1_postFit} \\
		\includegraphics[width=.45\textwidth]{Chapters/CH8/figures/SPLUSB_CRSR_UsingDL1rcFullSys/Plots/SR2} &
		\includegraphics[width=.45\textwidth]{Chapters/CH8/figures/SPLUSB_CRSR_UsingDL1rcFullSys/Plots/SR2_postFit} \\
	\end{tabular}
	\caption{Pre-fit (left) and post-fit (right) BDTG output distributions in SR1 and SR2 for the S+B \tZc fit in SRs+CRs with realistic Asimov.
		\ErrStatSys
	}%
	\label{fig:stat:tzc:splusb:crsr:srplots:1}
\end{figure}

\begin{figure}[htbp]
	\centering
	\begin{tabular}{cc}
		\includegraphics[width=.45\textwidth]{Chapters/CH8/figures/SPLUSB_CRSR_UsingDL1rcFullSys/Plots/SR3} &
		\includegraphics[width=.45\textwidth]{Chapters/CH8/figures/SPLUSB_CRSR_UsingDL1rcFullSys/Plots/SR3_postFit} \\
	\end{tabular}
	\caption{Pre-fit (left) and post-fit (right) leading lepton \pt distributions in SR3 for the S+B \tZc fit in SRs+CRs with realistic Asimov.
		\ErrStatSys
	}%
	\label{fig:stat:tzc:splusb:crsr:srplots:2}
\end{figure}

\begin{figure}[htbp]
	\centering
	\begin{tabular}{cc}
		\includegraphics[width=.45\textwidth]{Chapters/CH8/figures/SPLUSB_CRSR_UsingDL1rcFullSys/Plots/SBCR1} &
		\includegraphics[width=.45\textwidth]{Chapters/CH8/figures/SPLUSB_CRSR_UsingDL1rcFullSys/Plots/SBCR1_postFit} \\
		\includegraphics[width=.45\textwidth]{Chapters/CH8/figures/SPLUSB_CRSR_UsingDL1rcFullSys/Plots/SBCR2} &
		\includegraphics[width=.45\textwidth]{Chapters/CH8/figures/SPLUSB_CRSR_UsingDL1rcFullSys/Plots/SBCR2_postFit} \\
	\end{tabular}
	\caption{Pre-fit (left) and post-fit (right) BDTG output distributions in the side-band CRs for the S+B \tZc fit in SRs+CRs with realistic Asimov.
		\ErrStatSys
	}%
	\label{fig:stat:tzc:splusb:crsr:crplots:1}
\end{figure}

\begin{figure}[htbp]
	\centering
	\begin{tabular}{cc}
		\includegraphics[width=.45\textwidth]{Chapters/CH8/figures/SPLUSB_CRSR_UsingDL1rcFullSys/Plots/TTCR} &
		\includegraphics[width=.45\textwidth]{Chapters/CH8/figures/SPLUSB_CRSR_UsingDL1rcFullSys/Plots/TTCR_postFit} \\
		\includegraphics[width=.45\textwidth]{Chapters/CH8/figures/SPLUSB_CRSR_UsingDL1rcFullSys/Plots/TTZCR} &
		\includegraphics[width=.45\textwidth]{Chapters/CH8/figures/SPLUSB_CRSR_UsingDL1rcFullSys/Plots/TTZCR_postFit} \\
	\end{tabular}
	\caption{Pre-fit (left) and post-fit (right) leading lepton \pt distributions in the \ttbar and \ttZ CRs for the S+B \tZc fit in SRs+CRs with realistic Asimov.
		\ErrStatSys
	}%
	\label{fig:stat:tzc:splusb:crsr:crplots:2}
\end{figure}

\section{S+B fit in SRs+CRs with unblinded data}
\label{sec:stat:tzc:splusb:unb}
In this section, results with unblinded data are presented. The combined fit has been performed in the background 
control and signal regions with data under the signal+background hypothesis.\\
This test statistic is used to measure the compatibility of the observed data with the signal+background hypothesis 
and to make statistical inferences about $\mathrm{\mu}$, such as upper limits using the $\mathrm{CL_{s}}$ method~\cite{Junk:1999kv,Read:2002hq}.
Plots and tables shown in this section are the following:
\begin{itemize}
	\item The value of the post-fit normalisation parameter of the free floating background is shown in \Cref{fig:stat:tzc:splusb:crsr:norm_unb}.
	\item The list of the systematic shapes that are dropped from the fit for each sample and for each region is shown in \cref{fig:stat:tzc:splusb:crsr:pruning_unb}.
	\item The pull distributions of the all nuisance parameters can be seen in \Cref{fig:stat:tzc:splusb:crsr:np:instr_unb,fig:stat:tzc:splusb:crsr:np:model_unb} and \Cref{fig:stat:tzc:splusb:crsr:gamma_unb}.  
	\item The correlation matrix of the nuisance parameters is shown in \Cref{fig:stat:tzc:splusb:crsr:corrmatrix_unb}.
	%Red and blue plots for the top five ranked NPs are shown in a
	%dedicated appendix (\Cref{sec:app:fit:redblue:tzc:splusb:crsr}).
	\item The ranking of the nuisance parameters is shown in \Cref{fig:stat:tzc:splusb:crsr:ranking_unb}. 
	\item Event yields pre- and post-fit are shown in \Cref{tab:stat:tzc:splusb:crsr:yields:prefit_unb,tab:stat:tzc:splusb:crsr:yields:postfit_unb}. 
	\item Pre-fit and post-fit distributions of the fitted distributions in the various regions are shown in \Cref{fig:stat:tzc:splusb:crsr:srplots:1_unb,fig:stat:tzc:splusb:crsr:srplots:2_unb,fig:stat:tzc:splusb:crsr:crplots:1_unb,fig:stat:tzc:splusb:crsr:crplots:2_unb}.
\end{itemize}
The fake normalization factor $\mu_{\ttbar+\Wt}$ (\Cref{fig:stat:tzc:splusb:crsr:norm_unb}) is compatible with unity and the signal strength $\mu$ is compatible with zero within the uncertainties.\\
Results are similar to what was found with blinded Signal Regions (\Cref{sec:stat:tzc:splusb:crsr}).\\
The value of the fitted nuisance parameters (\Cref{fig:stat:tzc:splusb:crsr:np:instr_unb,fig:stat:tzc:splusb:crsr:np:model_unb}) are within their prior uncertainties,
meaning that the data are well modelled with the MC predictions within the uncertainties.\\
%The most pulled NPs (e.g. \ttbar FSR and \VVHF Berends scaling for events with 1 jet) are
%not among the highest ranked NPs (\Cref{fig:stat:tzc:splusb:crsr:ranking}). 
%Slightly pulled NPs (e.g. \ttZ normalization, Other fakes norm. and \VVHF generator) have a small post-fit impact on the signal strength, around 2\%, as can be seen again in \Cref{fig:stat:tzc:splusb:crsr:ranking}. Moreover, 
Concerning the correlations between NPs
(\Cref{fig:stat:tzc:splusb:crsr:corrmatrix_unb}), some strong correlations
between diboson related NPs are present, as expected. This is also
true for the \ttbar normalisation and some \ttbar modeling NPs. 
None of the systematic uncertainties has a post-fit impact on the signal strength parameter greater than 4\% as can be seen again in \Cref{fig:stat:tzc:splusb:crsr:ranking}.\\
Event yields pre- and post-fit, shown in \Cref{tab:stat:tzc:splusb:crsr:yields:prefit_unb,tab:stat:tzc:splusb:crsr:yields:postfit_unb} and distributions in \Cref{fig:stat:tzc:splusb:crsr:srplots:1_unb,fig:stat:tzc:splusb:crsr:srplots:2_unb,fig:stat:tzc:splusb:crsr:crplots:1_unb,fig:stat:tzc:splusb:crsr:crplots:2_unb} show a good agreement between the observed data and MC predictions in Control and Signal Regions. No evidence for the FCNC $\mathrm{\Pqt\rightarrow\PZ\Pqc}$ signal is found.

\begin{figure}[htbp]
	\centering
	\includegraphics[width=.5\textwidth]{Chapters/CH8/figures/SPLUSB_CRSR_DL1rc_unblind/NormFactors}
	\caption{Normalisation factors for the S+B \tZc fit in SRs+CRs with data.}%
	\label{fig:stat:tzc:splusb:crsr:norm_unb}
\end{figure}

\begin{figure}[htbp]
	\centering
	\includegraphics[width=.85\textwidth]{Chapters/CH8/figures/SPLUSB_CRSR_DL1rc_unblind/Pruning}
	\caption{Pruning of the nuisance parameters for the S+B \tZc fit in SRs+CRs with data.}%
	\label{fig:stat:tzc:splusb:crsr:pruning_unb}
\end{figure}

\begin{figure}[htbp]
	\centering
	\includegraphics[width=.8\textwidth]{Chapters/CH8/figures/SPLUSB_CRSR_DL1rc_unblind/NuisPar_Instrumental}
	\caption{Pulls and constraints of the instrumental nuisance parameters for the S+B \tZc fit in SRs+CRs with data.}%
	\label{fig:stat:tzc:splusb:crsr:np:instr_unb}
\end{figure}

\begin{figure}[htbp]
	\centering
	\includegraphics[width=.85\textwidth]{Chapters/CH8/figures/SPLUSB_CRSR_DL1rc_unblind/NuisPar_Theoretical_&_Modelling}
	\caption{Pulls and constraints of the theoretical and modeling nuisance parameters for the S+B \tZc fit in SRs+CRs with data.}%
	\label{fig:stat:tzc:splusb:crsr:np:model_unb}
\end{figure}

\begin{figure}[htbp]
	\centering
	\includegraphics[width=.85\textwidth]{Chapters/CH8/figures/SPLUSB_CRSR_DL1rc_unblind/Gammas}
	\caption{Gamma parameters for the S+B \tZc fit in SRs+CRs with data.}%
	\label{fig:stat:tzc:splusb:crsr:gamma_unb}
\end{figure}

\begin{figure}[htbp]
	\centering
	\includegraphics[width=.95\textwidth]{Chapters/CH8/figures/SPLUSB_CRSR_DL1rc_unblind/CorrMatrix}
	\caption{Correlation matrix of the nuisance paramenters for the S+B \tZc fit in SRs+CRs with data.}%
	\label{fig:stat:tzc:splusb:crsr:corrmatrix_unb}
\end{figure}

\begin{figure}[htbp]
	\centering
	\includegraphics[width=.9\textwidth]{Chapters/CH8/figures/SPLUSB_CRSR_DL1rc_unblind/Ranking}
	\caption{Ranking of the nuisance parameters for the S+B \tZc fit in SRs+CRs with data.}%
	\label{fig:stat:tzc:splusb:crsr:ranking_unb}
\end{figure}

\FloatBarrier
\clearpage
%\global\pdfpageattr\expandafter{\the\pdfpageattr/Rotate 90}
\begin{table}[]
	\centering
	\tiny
	% NB: add to main document: 
% \usepackage{siunitx} 
% \sisetup{separate-uncertainty,table-format=6.3(6)}  % hint: modify table-format to best fit your tables
\begin{tabular}{|p{0.10\textwidth}|>{\centering}p{0.08\textwidth}|>{\centering}p{0.08\textwidth}|>{\centering}p{0.08\textwidth}|>{\centering}p{0.09\textwidth}|>{\centering}p{0.09\textwidth}|>{\centering}p{0.09\textwidth}|>{\centering\arraybackslash}p{0.09\textwidth}|}
\toprule  
& {SR1} & {SR2} & {SR3} & {Side-band CR1} & {Side-band CR2} & {\ttZ CR} & {\ttbar CR}\\
\midrule 
    \ttZ+\tWZ   & 168 $\pm$ 22 & 33 $\pm$ 7 & 82 $\pm$ 11 & 88 $\pm$ 12 & 9.1 $\pm$ 2.1 & 164 $\pm$ 22 & 14.8 $\pm$ 1.9 \\ 
\ttW   & 5.8 $\pm$ 1.0 & 3.3 $\pm$ 0.6 & 2.04 $\pm$ 0.35 & 4.3 $\pm$ 0.7 & 2.5 $\pm$ 0.5 & 2.3 $\pm$ 0.5 & 27 $\pm$ 4 \\ 
\ttH   & 6.1 $\pm$ 1.0 & 0.88 $\pm$ 0.18 & 2.6 $\pm$ 0.4 & 2.3 $\pm$ 0.4 & 0.36 $\pm$ 0.07 & 5.4 $\pm$ 0.9 & 13.8 $\pm$ 2.1 \\ 
\VVLF   & 28 $\pm$ 17 & 35 $\pm$ 13 & 2.9 $\pm$ 2.0 & 25 $\pm$ 15 & 18 $\pm$ 7 & 0.20 $\pm$ 0.22 & 0.40 $\pm$ 0.21 \\ 
\VVHF   & 140 $\pm$ 100 & 160 $\pm$ 70 & 30 $\pm$ 22 & 130 $\pm$ 80 & 69 $\pm$ 28 & 13 $\pm$ 11 & 2.3 $\pm$ 1.4 \\ 
\tZq   & 47 $\pm$ 7 & 110 $\pm$ 18 & 13.8 $\pm$ 2.3 & 20 $\pm$ 4 & 9.9 $\pm$ 1.7 & 14.6 $\pm$ 2.9 & 0.90 $\pm$ 0.15 \\ 
\ttbar+Wt   & 21 $\pm$ 4 & 32 $\pm$ 11 & 3.7 $\pm$ 1.0 & 10 $\pm$ 4 & 9.1 $\pm$ 2.7 & 3.0 $\pm$ 1.2 & 102 $\pm$ 24 \\ 
Other fakes   & 10 $\pm$ 11 & 12 $\pm$ 12 & 1.4 $\pm$ 1.6 & 3 $\pm$ 5 & 10 $\pm$ 11 & 0.00 $\pm$ 0.06 & 0.12 $\pm$ 0.14 \\ 
Other   & 2.5 $\pm$ 1.5 & 3.8 $\pm$ 2.8 & 0.48 $\pm$ 0.25 & 2.2 $\pm$ 1.6 & 0.8 $\pm$ 2.6 & 1.1 $\pm$ 0.5 & 2.9 $\pm$ 1.5 \\ 
\midrule 
Total background  & 430 $\pm$ 110 & 390 $\pm$ 80 & 139 $\pm$ 25 & 280 $\pm$ 80 & 130 $\pm$ 32 & 203 $\pm$ 27 & 164 $\pm$ 25 \\ 
\midrule 
Data   & 433 & 443 & 143 & 331 & 169 & 197 & 156 \\ 
\midrule 
Data / Bkg.   & 1.00 $\pm$ 0.25 & 1.15 $\pm$ 0.24 & 1.03 $\pm$ 0.20 & 1.18 $\pm$ 0.35 & 1.30 $\pm$ 0.34 & 0.97 $\pm$ 0.14 & 0.95 $\pm$ 0.16 \\ 
\bottomrule 
\end{tabular} 

	\caption{Pre-fit event yields in the S+B \tZc fit in SRs+CRs with data. \TabErrStatSys} 
	\label{tab:stat:tzc:splusb:crsr:yields:prefit_unb}
\end{table} 

\begin{table}[]
	\centering
	\tiny
	% NB: add to main document: 
% \usepackage{siunitx} 
% \sisetup{separate-uncertainty,table-format=6.3(6)}  % hint: modify table-format to best fit your tables
\begin{tabular}{|l|S|S|S|S|S|S|S|}
\toprule  
 & {SR1tZc} & {SR2tZc} & {SR3tZc} & {Side-band CR1tZc} & {Side-band CR2} & {\ttZ CR} & {\ttbar CR}\\
\midrule 
  \ttZ+\tWZ   & 163 $\pm$  14 & 34 $\pm$  6 & 79 $\pm$  9 & 86 $\pm$  9 & 9.3 $\pm$  1.9 & 157 $\pm$  13 & 14.4 $\pm$  1.4 \\ 
  \ttW   & 5.7 $\pm$  0.9 & 3.4 $\pm$  0.6 & 2.0 $\pm$  0.4 & 4.2 $\pm$  0.7 & 2.5 $\pm$  0.5 & 2.2 $\pm$  0.4 & 26 $\pm$  4 \\ 
  \ttH   & 6.1 $\pm$  0.9 & 0.90 $\pm$  0.17 & 2.6 $\pm$  0.5 & 2.3 $\pm$  0.4 & 0.37 $\pm$  0.07 & 5.3 $\pm$  0.8 & 13.8 $\pm$  2.1 \\ 
  \VVLF   & 32 $\pm$  18 & 39 $\pm$  14 & 3.3 $\pm$  2.2 & 29 $\pm$  16 & 21 $\pm$  8 & 0.24 $\pm$  0.23 & 0.40 $\pm$  0.18 \\ 
  \VVHF   & 198 $\pm$  32 & 212 $\pm$  29 & 43 $\pm$  8 & 172 $\pm$  25 & 94 $\pm$  16 & 18 $\pm$  6 & 3.3 $\pm$  0.5 \\ 
  \tZq   & 46 $\pm$  7 & 112 $\pm$  16 & 13.9 $\pm$  2.4 & 19.6 $\pm$  3.3 & 10.1 $\pm$  1.6 & 14.4 $\pm$  2.5 & 0.91 $\pm$  0.12 \\ 
  \ttbar+Wt   & 18 $\pm$  4 & 30 $\pm$  7 & 3.2 $\pm$  0.8 & 9.5 $\pm$  2.8 & 8.6 $\pm$  1.7 & 2.5 $\pm$  0.8 & 95 $\pm$  13 \\ 
  Other fakes   & 15 $\pm$  11 & 17 $\pm$  13 & 2.1 $\pm$  1.7 & 5 $\pm$  5 & 18 $\pm$  15 & 0.005 $\pm$  0.009 & 0.18 $\pm$  0.13 \\ 
  Other   & 2.2 $\pm$  1.2 & 3.7 $\pm$  2.5 & 0.44 $\pm$  0.24 & 1.8 v 1.2 & 0.2 $\pm$  0.8 & 1.0 $\pm$  0.5 & 2.7 $pm$  1.4 \\ 
  FCNC (c)tZ   & 0.0 $pm$ 0.6 & 0.0 $pm$ 2.1 & 0.00 $pm$ 0.21 & 0.00 $pm$ 0.18 & 0.00 $pm$ 0.15 & 0.00 $pm$ 0.04 & 0.000 $pm$ 0.015 \\ 
  FCNC \ttbar(cZ)   & 0 $pm$ 10 & 0.1 $pm$ 3.2 & 0 $pm$ 4 & 0.0 $pm$ 0.7 & 0.01 $pm$ 0.34 & 0.0 $pm$ 0.6 & 0.00 $pm$ 0.06 \\ 
  \midrule 
  Total background  & 487 $pm$ 21 & 452 $pm$ 20 & 150 $pm$ 12 & 328 $pm$ 17 & 165 $pm$ 13 & 201 $pm$ 12 & 157 $pm$ 12 \\ 
  \midrule 
  Data   & 488 & 452 & 150 & 331 & 169 & 197 & 156 \\ 
  \midrule 
  Data / Bkg.   & 1.00 $pm$ 0.04 & 1.00 $pm$ 0.04 & 1.00 $pm$ 0.08 & 1.01 $pm$ 0.05 & 1.02 $pm$ 0.08 & 0.98 $pm$ 0.06 & 0.99 $pm$ 0.08 \\ 
  \bottomrule 
\end{tabular} 

	\caption{Post-fit event yields in the S+B\tZc fit in SRs+CRs with data. \TabErrStatSys} 
	\label{tab:stat:tzc:splusb:crsr:yields:postfit_unb}
\end{table} 
\clearpage
\FloatBarrier
%\global\pdfpageattr\expandafter{\the\pdfpageattr/Rotate 0}

\clearpage
\begin{figure}[htbp]
	\centering
	\begin{tabular}{cc}
		\includegraphics[width=.45\textwidth]{Chapters/CH8/figures/SPLUSB_CRSR_DL1rc_unblind/Plots/SR1} &
		\includegraphics[width=.45\textwidth]{Chapters/CH8/figures/SPLUSB_CRSR_DL1rc_unblind/Plots/SR1_postFit} \\
		\includegraphics[width=.45\textwidth]{Chapters/CH8/figures/SPLUSB_CRSR_DL1rc_unblind/Plots/SR2} &
		\includegraphics[width=.45\textwidth]{Chapters/CH8/figures/SPLUSB_CRSR_DL1rc_unblind/Plots/SR2_postFit} \\
	\end{tabular}
	\caption{Pre-fit (left) and post-fit (right) BDTG output distributions in SR1 and SR2 for the S+B \tZc fit in SRs+CRs with data.
		\ErrStatSys
	}%
	\label{fig:stat:tzc:splusb:crsr:srplots:1_unb}
\end{figure}

\begin{figure}[htbp]
	\centering
	\begin{tabular}{cc}
		\includegraphics[width=.45\textwidth]{Chapters/CH8/figures/SPLUSB_CRSR_DL1rc_unblind/Plots/SR3} &
		\includegraphics[width=.45\textwidth]{Chapters/CH8/figures/SPLUSB_CRSR_DL1rc_unblind/Plots/SR3_postFit} \\
	\end{tabular}
	\caption{Pre-fit (left) and post-fit (right) leading lepton \pt distributions in SR3 for the S+B \tZc fit in SRs+CRs with data.
		\ErrStatSys
	}%
	\label{fig:stat:tzc:splusb:crsr:srplots:2_unb}
\end{figure}

\begin{figure}[htbp]
	\centering
	\begin{tabular}{cc}
		\includegraphics[width=.45\textwidth]{Chapters/CH8/figures/SPLUSB_CRSR_DL1rc_unblind/Plots/SBCR1} &
		\includegraphics[width=.45\textwidth]{Chapters/CH8/figures/SPLUSB_CRSR_DL1rc_unblind/Plots/SBCR1_postFit} \\
		\includegraphics[width=.45\textwidth]{Chapters/CH8/figures/SPLUSB_CRSR_DL1rc_unblind/Plots/SBCR2} &
		\includegraphics[width=.45\textwidth]{Chapters/CH8/figures/SPLUSB_CRSR_DL1rc_unblind/Plots/SBCR2_postFit} \\
	\end{tabular}
	\caption{Pre-fit (left) and post-fit (right) BDTG output distributions in the side-band CRs for the S+B \tZc fit in SRs+CRs with data.
		\ErrStatSys
	}%
	\label{fig:stat:tzc:splusb:crsr:crplots:1_unb}
\end{figure}

\begin{figure}[htbp]
	\centering
	\begin{tabular}{cc}
		\includegraphics[width=.45\textwidth]{Chapters/CH8/figures/SPLUSB_CRSR_DL1rc_unblind/Plots/TTCR} &
		\includegraphics[width=.45\textwidth]{Chapters/CH8/figures/SPLUSB_CRSR_DL1rc_unblind/Plots/TTCR_postFit} \\
		\includegraphics[width=.45\textwidth]{Chapters/CH8/figures/SPLUSB_CRSR_DL1rc_unblind/Plots/TTZCR} &
		\includegraphics[width=.45\textwidth]{Chapters/CH8/figures/SPLUSB_CRSR_DL1rc_unblind/Plots/TTZCR_postFit} \\
	\end{tabular}
	\caption{Pre-fit (left) and post-fit (right) leading lepton \pt distributions in the \ttbar and \ttZ CRs for the S+B \tZc fit in SRs+CRs with data.
		\ErrStatSys
	}%
	\label{fig:stat:tzc:splusb:crsr:crplots:2_unb}
\end{figure}

\clearpage
\section{Results}
\label{sec:stat:tzc}
\paragraph {Blinded data} From the likelihood fit described in this chapter, namely the
signal+background fit in SRs+CRs with the realistic Asimov datasets in \Cref{sec:stat:tzc:splusb:crsr},
expected upper limits can be computed with the $\mathrm{CL_{s}}$ method~\cite{Junk:1999kv,Read:2002hq} with the expected 95\% confidence level (CL) limit on the branching ratio BR($\Pqt\rightarrow\PZ\Pqc$).\\
%The expected upper limit on $\mathrm{\mu}$ is the one which would be obtained if the data events
%were perfectly described by the expected background.
Table \ref{tab:limits:comparison} shows the expected limits on BR($t \rightarrow Zc$) extracted for various selections: 
\begin{itemize}
	\item without the use of any c-tagger ('Baseline'),
	\item using SMT, presented in \Cref{app:stat_smt:tzc:splusb:crsr},
	\item using \DLrc.
\end{itemize}
The improvement over baseline obtained using SMT is around 3\%, while using \DLrc is around 10\%.
\begin{table}[htbp]
	\centering
	\begin{tabular}{c|c|c}
		\toprule
		\multicolumn{3}{c}{Expected limits on BR($\Pqt\rightarrow\PZ\Pqc$) [$ \times 10^{-5}$] }\\
		\toprule
		\textbf{Baseline}          & \textbf{Using SMT}			& \textbf{Using \DLrc} \\
		\midrule
		$10.7 $ 	& $ 10.4 $   & $  9.6 $\\
		\bottomrule
	\end{tabular}
	\caption{ Expected limits on the branching ratios of $\Pqt\rightarrow\PZ\Pqc$. 
		Results using \DLrc, SMT and none of them are reported to estimate the impact of these techniques on the analysis.  }%
	\label{tab:limits:comparison}
\end{table}
\\The expected limits, using \DLrc, together
with statistical only limits and the expected limits from the previous ATLAS
analysis~\cite{TOPQ-2017-06}, are reported in
\Cref{tab:results:limits}.\\
The overall impact of systematics on the expected limit is 22\%.\\
The expected upper limit from the previous analysis is improved by a factor of 3.3 for the \tZc coupling.
\begin{table}[htbp]
	\centering
	\begin{tabular}{lccc} 
		\toprule
		\textbf{Limits} & \textbf{-1$\sigma [\times 10^{-5}]$} & \textbf{Expected  $[\times 10^{-5}]$} & \ \textbf{+1$\sigma [\times 10^{-5}]$}  \\
		\midrule
		BR ($\Pqt\rightarrow\PZ\Pqc$) \cite{TOPQ-2017-06} & 22 & 32& 46 \\
		BR ($\Pqt\rightarrow\PZ\Pqc$) (stat. only)                & 5.3 & 7.4 & 10.5 \\
		BR ($\Pqt\rightarrow\PZ\Pqc$)                                 & 6.9 & 9.6 & 13.8\\		
		\bottomrule
	\end{tabular}
	\caption{
		Expected upper limits on the branching ratios of $\Pqt\rightarrow\PZ\Pqc$.
		Expected upper limit from \cite{TOPQ-2017-06} is also included for reference.
	}%
	\label{tab:results:limits}
\end{table}
%WITH CALIBRATION
%root -l -q -b macros/getBR.C\(0.00024,0.299678\) 7.1910630e-05
%root -l -q -b macros/getBR.C\(0.00024,0.415899\) 9.9801765e-05
%root -l -q -b macros/getBR.C\(0.00024,0.603668\) 0.00014486654
% 6% improvement using dl1r wrt baseline

\paragraph {Unblinded data}
From the likelihood fit described in \Cref{sec:stat:tzc:splusb:unb} under the Signal+Background hypothesis, no evidence for the FCNC $\mathrm{\Pqt\rightarrow\PZ\Pqc}$ signal is found but a good agreement between data and Standard Model is observed. The results obtained under the Background-only hypothesis shows similar results (see Appendix \ref{app:stat:tzc:bonly:unb}).
In the absence of signal, the 95\% CL upper limit is set on $\mathrm{BR (\Pqt\rightarrow\PZ\Pqc)}$.
\Cref{fig:stat:tzc:splusb:crsr:CLsPlot} shows the observed and expected $\mathrm{CL_{s}}$ as a function of $\mathrm{BR (\Pqt\rightarrow\PZ\Pqc)}$. The observed limit is $\mathrm{BR (\Pqt\rightarrow\PZ\Pqc)}<11.8\times 10^{-5}$, inside the $\pm$1$\sigma$ band of the expected limit: $\mathrm{BR (\Pqt\rightarrow\PZ\Pqc)}<9.5\times 10^{-5}$.
The observed upper limit is improved by a factor of 2, while the expected upper limit is improved by a factor of 2.5 compared to the previous analysis.
\begin{figure}[htbp]
	\centering
	\includegraphics[width=.8\textwidth]{Chapters/CH8/figures/SPLUSB_CRSR_DL1rc_unblind/CLsPlot}
	\caption{$\mathrm{CL_{s}}$ vs $\mathrm{BR (\Pqt\rightarrow\PZ\Pqc)}$ plot. The median expected $\mathrm{CL_{s}}$ under the Signal+Background hypothesis
(black dashed line) is displayed along with the $\pm$1 and $\pm$2 standard deviations bands
(green and yellow, respectively). The solid red line at $\mathrm{CL_{s}= 0.05}$ denotes the threshold below
which the hypothesis is excluded at 95\% CL.}
	\label{fig:stat:tzc:splusb:crsr:CLsPlot}
\end{figure}

\begin{table}[htbp]
	\centering
	\begin{tabular}{lc} 
		\toprule
		 								& BR ($\Pqt\rightarrow\PZ\Pqc) [\times 10^{-5}]$  \\
		\midrule
		Observed \cite{TOPQ-2017-06} 	 & 24 \\
		Observed  				   					& 11.8  \\
		Expected -1$\sigma$   				  &   6.9 \\
		Expected                    				&  9.5 \\
	    Expected +1$\sigma$  				 &  13.8 \\		
		\bottomrule
	\end{tabular}
	\caption{
		Observed and expected 95\% CL upper limits on the FCNC $\Pqt\rightarrow\PZ\Pqc$ branching ratio. 
		The expected central value is shown together with the $\pm$1$\sigma$ bands.
		Observed upper limit from \cite{TOPQ-2017-06} is also included for reference.
	}%
	\label{tab:results:limits_unb}
\end{table}

