
% this file is called up by thesis.tex
% content in this file will be fed into the main document

% ----------------------- introduction file header -----------------------
\chapter*{Conclusions}
\label{chapter:Conclusions}
\markboth{CONCLUSIONS}{Conclusions}

% ----------------------- paths to graphics ------------------------

% the code below specifies where the figures are stored
\graphicspath{Chapters/CH9/figures/}

% ----------------------------------------------------------------------
% ----------------------- introduction content -------------------------
% ----------------------------------------------------------------------

In this thesis, the search for flavour-changing neutral-current 
couplings between the top-quark and the \PZ boson using $pp$ collision 
data recorded by the ATLAS detector at the LHC have been presented.
The data were recorded at a center-of-mass energy of \SI{13}{\TeV} and correspond to the full Run-2 dataset with an integrated luminosity of \SI{139}{\ifb}. \\
The analysis is performed to search for \ttbar events with one top quark decaying through the $\Pqt\rightarrow\PZ\Pqc$ channel and the other through the dominant Standard Model mode $\Pqt\rightarrow\PW\Pqb$, where only Z boson decays into charged leptons and leptonic W boson decays are considered as signal.\\
Different techniques have been used to improve the final expected upper limits.
\vspace{\baselineskip}
\\The first technique is the SMT, for the tagging of heavy-flavour jets. It exploits the $b \rightarrow \mu + X$, $b \rightarrow c \rightarrow \mu + X$ and $c \rightarrow \mu + X$ decay chains (with a total BR$\approx$ 20\%), by identifying muons reconstructed inside jets. 
\vspace{\baselineskip}
\\The second technique is the \DLrc discriminant that is used for charm-tagging. It is based on a deep feed-forward neural network (NN) and recently developed by the ATLAS collaboration.
\vspace{\baselineskip}
\\In order to extract the \tZc couplings, a binned maximum likelihood fit is performed using the MC templates for both signal and background predictions.
There is good agreement between the data and Standard
Model expectations, and overall impact of systematics on the expected limit is 22\%.\\
The estimated improvement using SMT is around 2\%, while the improvement using \DLrc is around 10\% compared with the 'Baseline', in which none of the two techniques has been used.\\
The most stringent expected upper limit at 95\% CL is in fact obtained using \DLrc:
\begin{equation*}
\mathrm{BR (\Pqt\rightarrow\PZ\Pqc)=\SI{9.6e-5}{}}
\end{equation*}
improving the previous ATLAS results by a factor of 3.3.
