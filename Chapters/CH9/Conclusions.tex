
% this file is called up by thesis.tex
% content in this file will be fed into the main document

% ----------------------- introduction file header -----------------------
\chapter*{Conclusions}
\label{chapter:Conclusions}
\markboth{CONCLUSIONS}{Conclusions}

% ----------------------- paths to graphics ------------------------

% the code below specifies where the figures are stored
\graphicspath{Chapters/CH9/figures/}

% ----------------------------------------------------------------------
% ----------------------- introduction content -------------------------
% ----------------------------------------------------------------------
\vspace{-1cm}
In this thesis, the search for flavour-changing neutral-current 
couplings between the top-quark and the \PZ boson using $pp$ collision 
data recorded by the ATLAS detector at the LHC has been presented.
The data were recorded at a center-of-mass energy of \SI{13}{\TeV} and correspond to the full Run-2 dataset with an integrated luminosity of \SI{139}{\ifb}. \\
The analysis searches for \ttbar events where one top quark decays through the $\Pqt\rightarrow\PZ\Pqc$ channel and the other through the dominant Standard Model mode $\Pqt\rightarrow\PW\Pqb$ with fully leptonic final state, where the Z boson decays into  charged leptons and leptonic W boson decays are considered as signal.
The FCNC process of production of a single top-quark in association with a Z boson is also considered.\\
Two different techniques to identify the c-jet have been tested to improve the final expected upper limits and to provide explicit charm identification in case of positive signal.
\vspace{\baselineskip}
\\The first technique is the Soft Muon Tagging technique, for the tagging of heavy-flavour jets. It exploits the $b \rightarrow \mu + X$, $b \rightarrow c \rightarrow \mu + X$ and $c \rightarrow \mu + X$ decay chains (with a total BR$\approx$ 20\%), by identifying muons reconstructed inside jets. 
\vspace{\baselineskip}
\\The second technique is the \DLrc discriminant that is used for charm-tagging. It is based on a deep feed-forward neural network (NN) and recently developed by the ATLAS collaboration.
\vspace{\baselineskip}
\\In order to extract the limit on \tZc couplings, a multivariate analysis has been used and a binned maximum likelihood fit is performed on the Boosted Decision Tree (BDT) output, using the Monte Carlo templates for both signal and background predictions.\\
There is good agreement between the data and the Monte Carlo expectations, 
and the overall impact of systematics on the expected limit is 22\%.\\
The estimated improvement using SMT is around 2\%, while the improvement using \DLrc is around 10\% compared with a scenario where none of the two techniques has been used.\\
The most stringent expected upper limit at 95\% CL is in fact obtained using the \DLrc technique:
\begin{equation*}
\mathrm{BR (\Pqt\rightarrow\PZ\Pqc)=\SI{9.6e-5}{}}
\end{equation*}
improving the previous ATLAS results by a factor of 3.3.
