% This file contains macros that can be called up from connected TeX files
% It helps to summarise repeated code, e.g. figure insertion (see below).

% insert a centered figure with caption and description
% parameters 1:filename, 2:title, 3:description and label
\newcommand{\figuremacro}[3]{
	\begin{figure}[htbp]
		\centering
		\includegraphics[width=1\textwidth]{#1}
		\caption[#2]{\textbf{#2} - #3}
		\label{#1}
	\end{figure}
}

% insert a centered figure with caption and description AND WIDTH
% parameters 1:filename, 2:title, 3:description and label, 4: textwidth
% textwidth 1 means as text, 0.5 means half the width of the text
\newcommand{\figuremacroW}[4]{
	\begin{figure}[htbp]
		\centering
		\includegraphics[width=#4\textwidth]{#1}
		\caption[#2]{\textbf{#2} - #3}
		\label{#1}
	\end{figure}
}

% inserts a figure with wrapped around text; only suitable for NARROW figs
% o is for outside on a double paged document; others: l, r, i(inside)
% text and figure will each be half of the document width
% note: long captions often crash with adjacent content; take care
% in general: above 2 macro produce more reliable layout
\newcommand{\figuremacroN}[3]{
	\begin{wrapfigure}{o}{0.5\textwidth}
		\centering
		\includegraphics[width=0.48\textwidth]{#1}
		\caption[#2]{{\small\textbf{#2} - #3}}
		\label{#1}
	\end{wrapfigure}
}

% predefined commands by Harish
\newcommand{\PdfPsText}[2]{
  \ifpdf
     #1
  \else
     #2
  \fi
}

\newcommand{\IncludeGraphicsH}[3]{
  \PdfPsText{\includegraphics[height=#2]{#1}}{\includegraphics[bb = #3, height=#2]{#1}}
}

\newcommand{\IncludeGraphicsW}[3]{
  \PdfPsText{\includegraphics[width=#2]{#1}}{\includegraphics[bb = #3, width=#2]{#1}}
}

\newcommand{\InsertFig}[3]{
  \begin{figure}[!htbp]
    \begin{center}
      \leavevmode
      #1
      \caption{#2}
      \label{#3}
    \end{center}
  \end{figure}
}


%----- NEWCOMMANDS ELIA ----------------------------

\newcommand{\sref}[1]{Sec.~\ref{#1}}
\newcommand{\fref}[1]{Fig.~\ref{#1}}
\newcommand{\aref}[1]{Appendix~\ref{#1}}
\newcommand{\figwidth}{0.9\columnwidth}

\newcommand{\argc}[1]{\left[#1\right]}
\newcommand{\arga}[1]{\left\lbrace #1\right\rbrace }
\newcommand{\argp}[1]{\left(#1\right)}
\newcommand{\valabs}[1]{\vert #1\vert}
\newcommand{\moy}[1]{\left\langle  #1 \right\rangle }
\newcommand{\moydes}[1]{\overline{#1}}

\newcommand{\Wbar}{ \mathclap{\phantom{W}\overline{\phantom{I}}} W}
\newcommand{\dbar}{d\mkern-6mu\mathchar'26 \!}
\newcommand{\deltabar}{\delta \mkern-8mu\mathchar'26}
\newcommand{\sigc}[1]{\argc{\sigma}(#1)}
\newcommand{\tho}{{\text{\thorn}} }

\newcommand{\gdD}{\mathcal{D}}
\newcommand{\gdO}{\mathcal{O}}
\newcommand{\gdH}{\mathcal{H}}
\newcommand{\gdR}{\mathbb{R}}
\newcommand{\gdC}{\mathbb{C}}
\newcommand{\gdN}{\mathbb{N}}
\newcommand{\gdF}{\mathcal{F}}


\newcommand{\gdHel}{\gdH_{\text{el}}}
\newcommand{\gdHdis}{\gdH_{\text{dis}}}
\newcommand{\gdHtil}{\widetilde{\mathcal{H}}}
\newcommand{\gdHelt}{\widetilde{\gdH}_{\text{el}}}
\newcommand{\gdHdist}{\widetilde{\gdH}_{\text{dis}}}


%---------------------------------
% For the personal comments
%\newcommand{\commperso}[1]{\textcolor{magenta}{#1}}
%\newcommand{\reftoput}[1]{\textcolor{cyan}{#1}}
\newcommand{\commperso}[1]{\textcolor{black}{#1}}
\newcommand{\reftoput}[1]{\textcolor{black}{#1}}

%\newcommand{\meevid}[1]{\textcolor{NavyBlue}{#1}}
\newcommand{\meevid}[1]{\textcolor{black}{#1}}
%\newcommand{\meevidbf}[1]{\textbf{#1}}
\newcommand{\meevidbf}[1]{#1}


\newcommand{\codecouleurperso}{
	\textcolor{Emerald}{Emerald}
	\textcolor{PineGreen}{PineGreen}
	\textcolor{BlueViolet}{BlueViolet}
	\textcolor{DarkOrchid}{DarkOrchid}
	\textcolor{NavyBlue}{NavyBlue}
	\textcolor{MidnightBlue}{MidnightBlue}
	\textcolor{OrangeRed}{OrangeRed}
	\textcolor{RoyalBlue}{RoyalBlue}
	\textcolor{RoyalPurple}{RoyalPurple}
	\textcolor{TealBlue}{TealBlue}	
	\textcolor{Violet}{Violet}	
	\textcolor{RubineRed}{RubineRed}	
	\textcolor{RedViolet}{RedViolet}	
	\textcolor{VioletRed}{VioletRed}	
	\textcolor{Periwinkle}{Periwinkle}	
}
% Emerald, PineGreen, BlueViolet, DarkOrchid, NavyBlue, MidnightBlue, OrangeRed, RoyalBlue, RoyalPurple, TealBlue, Violet, RubineRed, RedViolet, VioletRed, Periwinkle


%---------------------------------


\newcommand{\footnoteremember}[2]{
  \footnote{#2}
  \newcounter{#1}
  \setcounter{#1}{\value{footnote}}
}
\newcommand{\footnoterecall}[1]{
  \footnotemark[\value{#1}]
}



%%% Local Variables: 
%%% mode: latex
%%% TeX-master: "~/Documents/LaTeX/CUEDThesisPSnPDF/thesis"
%%% End: 
